%Dokumentklasse
\documentclass[a4paper,12pt]{scrreprt}
\usepackage[left=4cm,right=3cm, bottom=3cm, top=3cm]{geometry}
%\usepackage[onehalfspacing]{setspace}
% ============= Packages =============

% Dokumentinformationen
\usepackage[
	pdftitle={Evaluierung von Elm als Frontend für Webapplikationen},
	pdfsubject={},
	pdfauthor={Philipp Meißner},
	pdfkeywords={},	
	%Links nicht einrahmen
	hidelinks
]{hyperref}


% Standard Packages
\usepackage[utf8]{inputenc}
\usepackage[ngerman]{babel}
\usepackage[T1]{fontenc}
\usepackage{nameref}
\usepackage{graphicx, subfig}
\graphicspath{{img/}}
\usepackage{fancyhdr}
\usepackage{lmodern}
\usepackage{color}
\usepackage{enumitem}
% For inline code
\usepackage{xcolor,listings}
\usepackage{soul}
\usepackage{hhline}

% zusätzliche Schriftzeichen der American Mathematical Society
\usepackage{amsfonts}
\usepackage{amsmath}

%Abkürzungsverzeichnis
%\usepackage[printonlyused,withpage]{acronym}
\usepackage{acronym}
% Abbildungen nicht unter Fußnote!
%\usepackage[bottom]{footmisc}



%nicht einrücken nach Absatz
%\setlength{\parindent}{0pt}

% LsListing
% JS:
\lstdefinelanguage{JavaScript}{
  keywords={break, case, catch, continue, debugger, default, delete, do, else, finally, for, function, if, in, instanceof, new, return, switch, this, throw, try, typeof, var, void, while, with, console, log},
  morecomment=[l]{//},
  morecomment=[s]{/*}{*/},
  morestring=[b]',
  morestring=[b]",
  sensitive=true
}
\lstdefinelanguage{Elm}{
  keywords={let, in, debug, crash, module, exposing, import, port, as, filter, List, Bool, String, number, Number, Char, Model, Float, alias, type, Int, map, case, of, if, else, then, User, Food},
  morecomment=[l]{--},
  morecomment=[s]{\{-}{-\}},
  morestring=[b]',
  morestring=[b]",
  sensitive=true
}
\lstdefinelanguage{cpp}{
  keywords={if, else, cout, endl, string, return},
  morecomment=[l]{//},
  morecomment=[s]{/*}{*/},
  morestring=[b]',
  morestring=[b]",
  sensitive=true
}
\lstset{
  numbers=left,
  stepnumber=1,
  xleftmargin=1.3em,
  firstnumber=1,
  numberfirstline=true,
  showstringspaces=false
}
% Break the line a little earlier so that cites and whatnot don't overlap.
\emergencystretch=1em

% ============= Kopf- und Fußzeile =============
\pagestyle{plain}
%
\lhead{}
\chead{}
\rhead{\slshape \leftmark}
%%
\lfoot{}
\cfoot{}
\rfoot{\thepage}
%%
\renewcommand{\headrulewidth}{0.4pt}
\renewcommand{\footrulewidth}{0pt}

% ============= Package Einstellungen & Sonstiges ============= 
%Besondere Trennungen
\hyphenation{A-ppli-ka-tion Ja-va-Script i-ni-tia-len}

% ============= Dokumentbeginn =============

\begin{document}
%Seiten ohne Kopf- und Fußzeile sowie Seitenzahl
\pagestyle{empty}
\begin{center}
\begin{tabular}{p{\textwidth}}


\begin{center}
\centerline{\includegraphics[scale=0.13]{img/logo.png}}
\end{center}


\\

\begin{center}
\LARGE{\textsc{
Evaluierung von Elm als\\ Frontend für Webapplikationen
}}
\end{center}

\\


\begin{center}
\large{Fachhochschule Kiel}
\end{center}

\\

\begin{center}
\textbf{\Large{Bachelor-Thesis}}
%\textbf{\Large{DRAFT}}
\end{center}


\begin{center}
zur Erlangung des akademischen Grades\\
Bachelor of Science
\end{center}


\begin{center}
vorgelegt von
\end{center}

\begin{center}
\large{\textbf{Philipp Meißner}} \\
\small{Matrikelnummer: 922432}
\end{center}


\\

\\\\


\begin{center}
\begin{tabular}{lll}
\textbf{Erstprüfer:} & & Prof. Dr. Robert Manzke\\
\textbf{Zweitprüfer:} & &Prof. Dr. rer. nat. Carsten Meyer\\
\textbf{Abgabetermin:} & & 07.07.2016\\
\end{tabular}
\end{center}

\end{tabular}
\end{center}
\pagestyle{plain}
\pagenumbering{roman}
\clearpage
\phantomsection
\addcontentsline{toc}{chapter}{Abstract}
\chapter*{Abstract}
\label{sec:Abstract}
Motiviert durch die Versprechungen, die durch die funktionale Programmiersprache Elm gemacht werden \cite[Vgl.]{elm-no-runtime-errors}, ist das Ziel der vorliegenden Bachelor-Thesis, die Programmiersprache, welche in natives JavaScript kompiliert wird, für die Verwendung als Frontend für Webapplikationen anhand einer empirischen Untersuchung zu evaluieren.

Im Zuge dessen wurde eine \acl{SPA} in nativen Elm-Code überführt. Dabei wurde versucht weitestgehend die \acl{CSS}s und \acl{JS}-Dateien beizubehalten. Diese praktische Ausarbeitung wurde anhand unterschiedlichster Bewertungskriterien, wie beispielsweise die Zuverlässigkeit der Applikation, die Performanz einer Elm-Applikation und Portabilität zwischen unterschiedlichen Betriebssystemen und Browsern evaluiert. Ferner wurde die Dateigröße einer minimalen Elm-Applikation ausgewertet und anderen Frameworks gegenübergestellt. Darüber hinaus wurde der erzeugte Quellcode auf Wartbarkeit, Lesbarkeit, und Wiederverwendbarkeit untersucht.

Die praktische Ausarbeitung und Evaluation ergab, dass die Programmiersprache Elm die aufgestellten Kriterien in fast allen Punkten erfüllt. Lediglich die Interoperabilität von bestehenden \ac{JS}-Skripten in eine Elm-Applikation, sowie die Dateigröße einer minimalen Elm-Applikation erwiesen sich als nicht vollständig erfüllt. Die Interoperabilität war dahingehend nicht vollständig funktional, als dass es notwendig war Änderungen an den bestehenden \ac{JS}-Skripten anzufertigen, sowie die nicht praktikable Möglichkeit \ac{CSS}-Dateien nativ in Elm einzubinden. Der Elm-Compiler lieferte keinen Weg die Dateigröße der kompilierten Applikation zu minimieren. Stattdessen musste auf externe Werkzeuge zurückgegriffen werden. Abgesehen davon erwies sich eine Elm-Applikation als sehr performant, effizient und zuverlässig. Zusätzlich gibt es bereits eine Vielzahl an unterstützenden Werkzeugen zur Entwicklung mit Elm.

Die Ergebnisse erlauben die Schlussfolgerung, dass die Entwicklung mit Elm sehr empfehlenswert für Frontend-Entwickler ist. Das Typensystem in Elm erlaubt dem Elm-Compiler eine Vielzahl an Fehlern vorab zu finden. Ferner ist es für einen Entwickler nicht weiter notwendig die erzeugte Elm-Applikation auf allen Browsern separat testen zu müssen, da der Elm-Compiler die Applikation in natives \ac{JS} kompiliert, welches kompatibel mit dem ECMAScript-Standard ist. Daraus entsteht eine enorme Zeitersparnis, die gepaart mit einer effizienten Applikation und einem zuverlässigen Compiler einen deutlich positiven Eindruck von Elm hinterlässt.
%\pagenumbering{arabic}
% Beendet eine Seite und erzwingt auf den nachfolgenden Seiten die Ausgabe aller Gleitobjekte (z.B. Abbildungen), die bislang definiert, aber noch nicht ausgegeben wurden. Dieser Befehl fügt, falls nötig, eine leere Seite ein, sodaß die nächste Seite nach den Gleitobjekten eine ungerade Seitennummer hat. 
\cleardoublepage
\pagestyle{plain}

%Inhaltsverzeichnis
\tableofcontents

\clearpage
\phantomsection
\addcontentsline{toc}{chapter}{Abkuerzungsverzeichnis}
\chapter*{Abkürzungsverzeichnis}
\label{sec:Abkürzungsverzeichnis}


%Abkürzungsverzeichnis
%\section{Abkürzungsverzeichnis}
\begin{acronym}[ICNDb ]
\acro{API}{Application Programming Interface}
\acro{APP}{Applikation}
\acro{CSS}{Cascading Style Sheet}
\acro{DOM}{Document Object Model}
\acro{HTML}{HyperText Markup Language}+
\acro{HTTP}{Hypertext Transfer Protocol}
\acro{ICNDb}{The Internet Chuck Norris Database}
\acro{ID}{Identifier}
\acro{IDE}{Integrated Development Environment}
\acro{JS}{JavaScript}
\acro{JSON}{JavaScript Object Notation}
\acro{MVU}{Model-View-Update}
\acro{NPM}{Node Package Manager}
\acro{REPL}{Read-Evaluate-Print-Loop}
\acro{SPA}{Single Page Application}
\acro{URL}{Uniform Resource Locator}
\end{acronym}

%Verzeichnis aller Bilder
\clearpage
\phantomsection
\addcontentsline{toc}{chapter}{Abbildungsverzeichnis}
\listoffigures
%Verzeichnis aller Tabellen
%\listoftables

% pagestyle für gesamtes Dokument aktivieren
\cleardoublepage
\pagestyle{plain}
\pagenumbering{arabic}
\chapter{Einleitung}
\label{sec:einleitung}
Die Entwicklung von Webapplikationen kostet Zeit und dementsprechend Geld. Doch mit der Veröffentlichung der Webseite ist es nicht getan. Oftmals werden Fehler in der Webapplikation erst zur Laufzeit bemerkt und müssen daraufhin behoben werden. Wieder entstehen Kosten, nicht nur durch den Zeitaufwand der erneuten Entwicklung, sondern auch durch mögliche Ausfälle der Webapplikation. Die neue funktionale Programmiersprache Elm verspricht Besserung, indem Laufzeitfehler der Vergangenheit angehören sollen \cite[Vgl. f.]{elm-no-runtime-errors}. Dies soll nicht zuletzt an dem eigens entwickelten Compiler liegen, der den geschriebenen Quellcode überprüft und eine immense Anzahl von Fehlern finden soll. Zusätzlich sollen die Fehlermeldungen sehr präzise und aussagekräftig sein, wodurch dem Entwickler bei der Programmierung zusätzlich Zeit erspart wird. Darüber hinaus gilt die dargestellte Applikation als sehr performant.
Im Rahmen dieser wissenschaftlichen Arbeit sollen diese Aussagen überprüft werden. Zu diesem Zweck werden in Kapitel \ref{sec:grundlagen} \glqq\nameref{sec:grundlagen}\grqq~notwendige Grundlagen geklärt. Unter anderem behandelt das Kapitel das Konzept funktionaler Programmiersprachen, gefolgt von einer Einführung in die Grundlagen der Programmiersprache Elm. Hier werden  Im Anschluss werden in Kapitel \ref{sec:Evaluierung der Programmiersprache Elm} \glqq\nameref{sec:Evaluierung der Programmiersprache Elm}\grqq~Bewertungskriterien vorgestellt und erläutert, anhand derer die vorherigen Aussagen überprüft werden. Dazu gehören unter anderem die Evaluation der Performanz, die Klärung der Frage wie hilfreich der Elm-Compiler tatsächlich ist und ob mit der Verwendung von Elm Kompromisse bei der Kompatibilität mit unterschiedlichen Browsern gemacht werden müssen. Dem folgt die Dokumentation und Erläuterung des praktischen Teils dieser wissenschaftlichen Arbeit, in der zunächst einmal der geplante Programmablauf, sowie die genutzte Entwicklungsumgebung vorgestellt werden.  und mündet in einer Auswertung anhand der aufgestellten Bewertungskriterien. Auf Grundlage der Auswertung werden im letzten Kapitel \ref{chap:fazit} \glqq\nameref{chap:fazit}\grqq~die Ergebnisse gegenübergestellt. Das Ziel dieser wissenschaftlichen Arbeit soll es sein, die Daseinsberechtigung der Programmiersprache Elm im Bereich der Frontend\footnote{Bezeichnet weitestgehend den Teil einer Webapplikation, der dem Nutzer dargestellt wird.}-Webentwicklung zu prüfen und eine Aussage darüber zu treffen, ob Elm die herkömmliche Entwicklung von Webapplikationen verbessert und tatsächlich eine valide Alternative darstellt.

%Die heutige Welt ist sehr schnelllebig und spielt sich immer mehr im Internet ab. Firmen präsentieren sich auf ihren Webseiten und akquirieren dadurch Neukunden. Ganze Geschäfte leben nur noch durch den Online-Handel und besitzen keinerlei Verkaufsläden, in denen ein Kunde das Produkt vorab in den Händen halten kann. Umso wichtiger ist es, das Produkt auf der Webseite außergewöhnlich gut zu präsentieren, um den Kunden zu überzeugen. Webseiten dieser und im Grunde jeglicher Art zielen immer darauf ab, einem Nutzer Informationen bereitzustellen und entsprechend angenehm zu präsentieren. Doch die Entwicklung solcher Systeme ist komplex und geht weit über das Design hinaus. Folglich ist ein großer Teil der Ausgaben von Firmen, die sich online präsentieren, die Bezahlung von Entwicklern für ihre Webapplikationen. Dies können simple Webseiten ohne großartige Funktionen sein, jedoch auch komplexe Systeme wie ein automatisierter Online-Handel, in dem die Nutzer ihren gesamten Einkauf abwickeln können, mitsamt Bezahlung. Demzufolge ist es für die Firmen von großem Interesse, dass die angestellten Entwickler zügig Ergebnisse in der Entwicklung der Webapplikationen machen. Damit der Entwickler effektiv arbeiten kann, braucht er Systeme, die ihn unterstützen. Angefangen bei den Werkzeugen wie seiner Entwicklungsumgebung, bis hin zur tatsächlichen Programmiersprache. Diese wissenschaftliche Arbeit befasst sich mit der neuen Programmiersprache Elm, welche eine funktionale, den deklarativen Programmierparadigmen folgende, Programmiersprache ist. Sie wurde zu Beginn ihrer Entwicklung für die Erstellung von grafischen Benutzeroberflächen und der Verbildlichung mathematischer Funktionen genutzt, bewegt sich nun jedoch immer weiter in Richtung der Webentwicklung und kommt mit einigen Neuheiten, Veränderungen und einer aktiven Gemeinschaft an Open-Source Entwicklern.
%
%Elm verspricht eine blitzschnelle Darstellung von Inhalten selbst bei riesigen Datenmengen mit Hilfe einer Technik die $virtual-dom$ genannt wird. Auch soll es keinerlei Laufzeitfehler mehr geben, da die gesamte Sprache mit Garantien ausgeschmückt ist, die im Zusammenspiel mit dem eigens entwickelten Compiler alle möglichen Fehler vorab findet und darauf hinweist.
%All diese Versprechungen werden anhand mehrerer Bewertungskriterien während einer empirischen Entwicklung einer Webapplikation geprüft.

\chapter{Theoretische Grundlagen}
\label{sec:grundlagen}
%TODO: Einleitung finden

\section{Funktionale Programmiersprachen}
\label{sec:funktionaleProgrammiersprache}
Jede Programmiersprache lässt sich in eine Kategorie, einem Programmierparadigma, einteilen.
Prof. Dr. Uwe Kastens beschreibt Programmierparadigmen als \glqq unterschiedliche Prinzipien, mit denen die Ausführung von Programmen beschrieben wird\grqq\cite{definition-programmierparadigmen}.
Ferner bezeichnet er die imperative, objektorientierte, funktionale und logische Programmierung als am Wichtigsten\footnote{\cite[Vgl.]{definition-programmierparadigmen}}. Jedes Programmierparadigma birgt unterschiedliche Merkmale der Programmierung, wie beispielsweise rekursive Funktionen, Funktionen höherer Ordnung oder Polymorphie. Es ist nicht ausgeschlossen, dass eine Programmierprache mehreren Prinzipien folgt und unterschiedliche Merkmale vereint. Funktionale Programmiersprachen sind stark an die Mathematik angelehnt, da sie sich syntaktisch sehr ähnlich sind. Programme werden hier eher als mathematische Funktion formuliert, die eingehende Argumente immer gleichermaßen zu einem Ergebnis auswerten.
\begin{figure}[h]
\begin{lstlisting}[language=JavaScript]
iterative_function(n) {
  sum = 0;
  for(i = 0; i <= n; i++) {
    sum += i;
  }
}
console.log(iterative_function(10)); // => 55
\end{lstlisting}
\caption{Eine iterative Funktion}\label{fig:iterative-function}
\end{figure}
Bei funktionalen Programmiersprachen ist es üblich, dass eine Variable nach ihrer Initialisierung ihren zugewiesenen Wert für die gesamte Laufzeit des Programmes beibehält. Man spricht hier von der Unveränderlichkeit einer Variable. Dadurch kann eine Variable als fester Ausdruck angesehen werden, der immer denselben Wert liefert.
Typischerweise gibt es in funktionalen Programmiersprachen keine Schleifen, da dies bereits eine Verletzung der Unveränderlichkeit von Variablen bedeuten würde, wie klar erkennbar bei der Iteration in Abbildung~\ref{fig:iterative-function} ist.
Hier wird bei jedem Durchlauf der Schleife die Variable $i$ inkrementiert und mit dem neuen Wert versehen. Es ist allerdings auch möglich eine Schleife in einer funktionalen Programmiersprache zu verwirklichen. In Abbildung~\ref{fig:recursive-function} ist die zuvor gezeigte iterative Funktion als rekursive Funktion implementiert worden.
\begin{figure}[h!]
\begin{lstlisting}[language=JavaScript]
recursive_function(n) {
  if (n < 1) return 0
  else return n + recursive_function(n-1);
}
console.log(recursive_function(10)); // => 55
\end{lstlisting}
\caption{Eine rekursive Funktion}\label{fig:recursive-function}
\end{figure}
\begin{quote}
\glqq Eine Funktion heißt rekursiv, wenn sie sich selbst direkt aufruft oder wenn ein Aufruf einer anderen von ihr aufgerufenen Funktion dazu führt, dass sie ihrerseits wieder aufgerufen wird.\grqq\cite{rekursive-funktion}
\end{quote}
In beiden Fällen ist das Ergebnis gleich, jedoch besitzt die rekursive Implementierung keinerlei Seiteneffekte\footnote{Ein Seiteneffekt beschreibt die mehrmalige Zuweisung einer Variable mit einem Wert.}. Es wird an keiner Stelle der Wert einer Variable verändert, es werden lediglich Aufrufe mit neuen Werten durchgeführt. Die iterative Implementierung verursacht hingegen zwei Seiteneffekte. Es wird sowohl die Variable $sum$, sowie $i$ bei jedem Schleifendurchlauf überschrieben. Bei der Rekursion hingegen wird  die Funktion $recursive\_function$
mit einem neuen Wert aufgerufen, ohne jemals die ursprüngliche Variable verändert zu haben.
Üblicherweise werden Funktionen in funktionalen Programmiersprachen als Funktionen höherer Ordnung angesehen. Das heißt, dass eine Funktion eine andere Funktion als Argument entgegen nehmen kann, oder eine Funktion als Rückgabewert hat. Diese Art von Funktionen ist bekannt unter dem Begriff $Lambda\,Funktion$ oder $anonyme\,Funktion$. Eine solche Funktion hat entsprechend keinen Namen, sondern kann nur einmalig aufgerufen werden. Im folgenden Beispiel sehen wir den Aufruf einer anonymen Funktion in Elm:
\begin{equation} \label{eq:solve}
(\backslash x \rightarrow x * 2)\ [ 2, 3, 4 ] \Longrightarrow [4, 6, 8]
\end{equation}
Grundsätzlich wird mit einer solchen Lambda-Funktion ein mathematisches Abbildungsgesetz formuliert. Das mathematische Gegenstück zu Funktion~\ref{eq:solve} ist $x \rightarrow x * 2$ und beschreibt eine Funktion, die einen Eingabeparameter $x$ auf $x*2$ abbildet. Hier wird klar erkenntlich, wie nahe die funktionale Schreibweise dem mathematischen Abbildungsgesetz ist. Auch Elm gehört dem funktionalen Programmierparadigma an und vereinheitlicht dessen Konzepte.

\section{Grundlagen der Programmiersprache Elm}
\label{sec:grundlagen-programmiersprache-elm}

\subsection{Geschichte}
\label{sec:Geschichte}
Elm wurde im Rahmen der Bachelorarbeit \glqq Elm: Concurrent FRP for Functional GUIs\grqq von Evan Czaplicki entwickelt. \cite[Vgl.]{evan-thesis}
Auslöser für die Entwicklung von Elm war für Czaplicki die Schwierigkeit, ein Bild auf einer Webseite sowohl horizontal, als auch vertikal zu zentrieren.\footnote{\cite[Vgl.]{evan-center}} Es gab keine für ihn annehmbare, leichte Lösung für dieses Unterfangen, ohne damit weitere Probleme zu schaffen. Unter der Leitfrage „What would web programming look like if we could restart?“, also „Wie würde Web-Programmierung aussehen, wenn wir neu starten könnten?“, machte er sich Gedanken, welche Veränderungen an den aktuellen, etablierten Programmiersprachen für die Webentwicklung wünschenswert wären und entwickelte den ersten Prototypen von Elm.

\subsection{Konzept}
\label{sec:Konzept}
Elm verfolgt eine ganz eigene Implementierung des Model-View-Controller Paradigmas. Hier wird es \ac{MVU} genannt. Anhand des Beispiels in Abbildung~\ref{fig:counter} lässt sich das Muster in drei grundlegende Stationen unterteilen und erklären.
\begin{figure}[h]
	\centering  
	\includegraphics[scale=0.5]{img/counter.png}
	\caption{Ein simpler Zähler auf einer Webseite}\label{fig:counter}
\end{figure}
Die Abbildung~\ref{fig:counter} zeigt einen simplen Zähler der über zwei Knöpfe inkrementiert und dekrementiert werden kann. Der aktuelle Stand des Zählers wird zwischen den Knöpfen angezeigt und kann sowohl negative, als auch positive Werte annehmen.
Der angezeigte Zählerwert ist das sogenannte $Model$ und zeigt den aktuellen Status der Applikation an. Interagiert ein Nutzer nun mit einem der beiden Knöpfe um den Zähler zu erhöhen oder zu reduzieren, wird diese Aktion an die sogenannte $Update$-Funktion weitergegeben. Zusätzlich zur auszuführenden $Aktion$, bekommt diese Funktion auch noch das aktuelle $Model$ übergeben.
Die $Update$-Funktion nimmt sämtliche Einwirkungen durch den Nutzer von außen entgegen und wendet diese Aktionen auf das aktuelle $Model$ an. Dabei wird jedoch nicht das $Model$ direkt verändert, sondern ein neues $Model$ mit den geänderten Werten zurückgegeben, da sonst ein Seiteneffekt die Folge wäre. Damit dieser Vorgang zügig vonstatten geht, nutzt Elm persistente Datenstrukturen, womit nur die tatsächlich geänderten Attribute eines Models im neuen $Model$ gesetzt werden, die unveränderten Attribute hingegen werden übernommen.
Das Ergebnis der Update-Funktion wird weitergereicht an die $View$-Funktion. Sie beschreibt das Aussehen der Webseite und kümmert sich um die Darstellung des $Model$s.
Der zuvor beschriebene Datenfluss in Elm wird noch einmal in Abbildung~\ref{fig:elm-model-view-update-concept} visualisiert.
\begin{figure}[h]
  \centering  
  \includegraphics[scale=1]{img/elm-model-view-update-concept.png}
  \caption{Das Model-View-Update Konzept von Elm}\label{fig:elm-model-view-update-concept}
\end{figure}

\subsection{Umsetzung}
\label{sec:Umsetzung}
Der vom Programmierer verfasste Programmcode wird vor der endgültigen Nutzung zu \ac{JS}, \ac{HTML} und \ac{CSS} kompiliert und in die Webseiten integriert. Entsprechend fungiert in Elm verfasster Code im Endeffekt wie natives \ac{JS}, nutzt allerdings noch einige weitere vertiefende Konzepte, um viele Problematiken von \ac{JS} zu umgehen und auszumerzen.
Unter anderem verspricht Elm, dass generierter Code keinerlei Laufzeitfehler erzeugt.\footnote{\cite[Vgl.]{elm-no-runtime-errors}} Sämtliche Fehlerquellen werden vom Compiler zuvor erkannt, abgefangen und an den Programmierer weitergeleitet, um sie zu beheben. Damit dies funktioniert, implementiert Elm unter anderem Konzepte des funktionalen Programmierparadigmas. Diese und die weiteren Grundkonzepte werden im folgenden erläutert.

\subsubsection{Keine Seiteneffekte}
\label{sec:Keine Seiteneffekte}
In Elm sind Seiteneffekte ausgeschlossen. Sämtliche Variablen sind unveränderlich und können nur einmalig mit einem Wert initiiert werden. Danach bleibt diese Variable bis zum Ende der Laufzeit unverändert. Wie bereits beschrieben gibt es in Elm das $Model$, welches die Informationen über den Status der Applikation enthält. Beschreibt das $Model$ beispielsweise den aktuellen Stand eines Zählers und wird dieser erhöht, muss auch das $Model$, um aktuell zu bleiben, verändert werden. Hier käme es zu einem Seiteneffekt. Realisiert wird diese Veränderung dadurch, dass ein neues $Model$ mit den gleichbleibenden Daten, sowie dem zu ändernden, aktualisierten Wert erstellt wird. Da ein neues $Model$ erstellt wurde, gibt es nun keinen Seiteneffekt mehr. Das vorherige $Model$ wird schlichtweg verworfen und mit einem neuen $Model$ ersetzt.
Das Konzept der unveränderbaren Werte wurde aus der Mathematik übernommen.  Um Funktionen und ihre Korrektheit garantieren zu können, wird dort dasselbe Prinzip der unveränderlichen Variablen angewandt. Betrachtet man beispielsweise den Ausdruck~\ref{eq:imperative_equation}, so fällt auf, dass die Schreibweise lediglich in den meisten imperativen Programmiersprachen sinnvoll ist, allerdings einen Seiteneffekt darstellt.
\begin{equation} \label{eq:imperative_equation}
x = x + 1
\end{equation}
In einer imperativen Programmiersprache wird durch den Ausdruck~\ref{eq:imperative_equation} der aktuellen Wert in $x$ ausgelesen, um $1$ inkrementiert und das Ergebnis in die Variable $x$ geschrieben.
Mathematisch betrachtet ist diese Aussage jedoch schlichtweg falsch, denn es existiert kein $x$, welches diese Aussage wahr werden lässt, wie durch Abbildung~\ref{eq:mathematical_equation} deutlich wird.
\begin{equation} \label{eq:mathematical_equation}
x=x+1 \leftrightarrow 0=1
\end{equation}
Die meisten imperativen Programmiersprachen nutzen das rechtsassoziative Gleichheitszeichen als Zuweisung, während es in der Mathematik als Vergleichsoperator angesehen wird.
Die eigentliche Bedeutung des Ausdrucks~\ref{eq:imperative_equation} ist mathematisch ausgedrückt:
\begin{equation} \label{eq:true_equation}
x_1:= x_0 + 1
\end{equation}
Es ist klar erkennbar, dass $x_1$ und $x_0$ nicht dieselbe Variable sind wodurch die Aussage nun als wahr eingestuft werden kann. Das beschriebene Konzept wird referentielle Transparenz genannt und beschreibt die Kontinuität des Wertes einer Variable.
Des Weiteren basieren Funktionen in Elm auf dem Konzept von reinen Funktionen($pure\,functions$). Das bedeutet, dass eine Funktion stets das gleiche Ergebnisse liefert, insofern auch die Eingabeparameter gleich bleiben, unabhängig vom Zeitpunkt der Ausführung. Beispiele für eine reine Funktion sind $sin(x)$ oder $add(x, y)$. Sie berechnen immer dieselben Werte, völlig unabhängig davon, wie oft oder zu welchem Zeitpunkt sie ausgeführt werden. Ein beliebtes Gegenspiel ist die $random()$ Funktion, die einen (semi-)zufälligen Wert zurückliefert und somit als eine unreine Funktion gilt. Doch auch diese Funktion kann als reine Funktion implementiert werden, wenn sie einen Wert abhängig von einem Übergabeparameter berechnet wie beispielsweise $random(seed)$.

\subsubsection{Elm-Compiler}
\label{sec:Elm-Compiler}
Laufzeitfehler sollen mit Elm in Vergessenheit geraten. Dafür soll der integrierte Compiler sorgen. Die Fehlermeldungen des Compilers sind sehr strikt und deuten exakt auf die Programmzeile die für den jeweiligen Fehler verantwortlich ist. Bei der herkömmlichen Entwicklung eines Frontends mit \ac{JS} trifft man häufig auf den Wert $undefined$. Dieser Wert beschreibt, dass die dazugehörige Variable noch nicht initialisiert wurde und somit keine nutzbaren Daten enthält. Trifft man nun auf diesen Wert und versucht eine andere Funktion darauf auszuführen, so kann das geschriebene Programm entsprechend abstürzen. Der Elm Compiler überprüft den Programmcode nach exakt diesen Situation beziehungsweise analysiert, ob Variablen und Funktionen vorab initialisiert wurden, welche Parameter die einzelnen Funktionen erwarten und ob die Rückgabeparameter dem Typen entsprechen, den die anderen Funktionen erwarten. Befolgt man die Anweisungen des Compilers, soll das in einem stark strukturierten, lesbaren und funktionieren Code münden.
Dem Programmierer werden weniger Möglichkeiten gegeben, bestimmte Ziele zu erreichen, doch dadurch soll auf lange Sicht einheitlicher, lesbarer und besser zu wartender Code erzeugt werden. Passend dazu gibt es bereits viele Erweiterungen für gängige Editoren wie Sublime Text und Atom, welche beim Speichern des Projektes den Code entsprechend des Style Guides strukturieren und formatieren. So kann einerseits die Lesbarkeit des Codes vereinheitlicht, andererseits die Fehlerquellen in Form von Einrückungsfehlern oder vergessenen Kommas verringert werden.

\subsubsection{Statische Typisierung}
\label{sec:Statische Typisierung}
Anders als bei nativem \ac{JS}, gibt es in Elm keine dynamische Typisierung. Das bedeutet, dass sowohl die Typen einer Variable, als auch die Rückgabewerte von Funktionen bereits bei der Kompilierung bekannt sein müssen. Natives \ac{JS} erlaubt es, dass die Typen von Variablen erst zur Laufzeit überprüft werden und sich zusätzlich in dieser Zeit ändern können. So ist der Quellcode in Abbildung~\ref{fig:dynamische-typisierung} konform in \ac{JS}.
\begin{figure}[ht]
\begin{lstlisting}[language=JavaScript]
var i = 1;
i = "Test";
\end{lstlisting}
\caption{Beispiel der dynamischen Typisierung}\label{fig:dynamische-typisierung}
\end{figure}
Der Typ der Variable $i$ wurde in der Abbildung~\ref{fig:dynamische-typisierung} während der Laufzeit von $number$ zu $string$ geändert. Da Elm stark typisiert ist, gibt es keine Möglichkeit, dass eine Funktion verschiedene Datentypen zurück gibt oder eine Variable mehrere Typen während der Laufzeit annimmt.


\subsubsection{Performanz}
\label{sec:Performanz}
%\begin{figure}[hb]
%  \centering  
%  \includegraphics[scale=0.6]{img/virtual-dom.png}
%  \caption{Performanz von Elm im Vergleich zu anderen Web-Frameworks. \cite{elm-performance}}\label{fig:performance}
%\end{figure}
Obwohl Daten nicht verändert, sondern durchgehend ein neuer, aktualisierter Datensatz erstellt wird, soll die Performanz nicht darunter leiden. Möglich soll das durch die Verwendung einer virtuellen \ac{DOM} werden.
Dabei wird das echte \ac{DOM} bei jedem „Frame“ in eine abstrakte Version kopiert. Auf diese abstrakte Version werden die Änderungen angewandt. Zunächst klingt diese Vorgehensweise sehr langsam und aufwändig, um jedoch die Geschwindigkeit zu gewährleisten wird das aktuelle abstrakte \ac{DOM} mit dem neuen, veränderten \ac{DOM} verglichen und nach Unterschieden gesucht. Jede Unterschiedlichkeit wird daraufhin zu einer Liste hinzugefügt, in der sämtliche Änderungen festgehalten werden. Anschließend wird diese Liste an den Browser zurückgegeben, so dass alle Änderungen für den Nutzer sichtbar gemacht werden können. Daraus resultiert, dass nur noch die tatsächlich neuen oder veränderten Elemente im \ac{DOM} des Nutzers aktualisiert werden müssen. Dieser zu verändernde Teil stellt nur einen Bruchteil des kompletten \ac{DOM} dar. Man kann dadurch von einer immensen Effizienzsteigerung ausgehen.

\subsubsection{Interoperabilität}
\label{sec:Interoperabilität}
Es wäre sehr aufwendig die gängigsten JavaScript-Bibliotheken und -Frameworks wie jQuery oder AngularJS komplett zu verwerfen und mit Elm zu realisieren respektive neu zu programmieren. Glücklicherweise bietet Elm eine ausgereifte Interoperabilität, wodurch alle Garantien der funktionalen Programmierung übernommen werden können, selbst wenn die externen Bibliotheken diese nicht gewährleisten.
Oftmals sind externe Bibliotheken nicht funktional programmiert, weswegen keinerlei Garantien seitens Elm gemacht werden können. Jedoch gibt Elm an, dass der Datenaustausch über isolierte Ports in beide Richtungen funktioniert.

%\subsubsection{Debugger}
%\label{sec:Debugger}
%TODO: Wurde in der letzten Version entfernt.
%Nicht nur die detaillierten Fehlermeldungen des Compilers sind ein großer Pluspunkt von Elm, sondern auch der mitgelieferte und einfach zu bedienende Debugger. Anders als bei anderen Programmiersprachen zeigt der Debugger nicht nur einen einfachen Stacktrace mit den letzten Rücksprungadressen an. Vielmehr ermöglicht es der Debugger in Elm die Zeit zurückzudrehen. In gängigen Debuggern ist es nicht einfach, ein bestimmtes Verhalten, das möglicherweise zum Absturz des Programms geführt hat, zu reproduzieren. Um den Fehler erneut zu erhalten und dadurch ein Verständnis des Fehlers aufzubauen ist es oft notwendig den genauen Hergang und Ablauf manuell nachzuahmen. Der Elm Debugger hingegen speichert den Status einer jeden Variable zu jedem Zeitpunkt und zeichnet außerdem die Wechsel auf. Aufgrund dessen ist es sehr einfach möglich mit Hilfe eines Sliders auf der Website die Zeit buchstäblich zurückzudrehen und den vorherigen Status wieder aufzurufen. Weiterhin besteht die Möglichkeit sämtliche Daten zu diesem Zeitpunkt zu betrachten und live zu verändern. Dadurch wird nicht nur der aktuell begutachtete Status verändert und der Effekt zeitgleich auf dem Bildschirm angezeigt, sondern auch alle folgenden Status mitsamt den Daten werden entsprechend aktualisiert. Evan Czaplicki hat dahingehend ein Video angefertigt, dass den Vorgang sehr detailliert beschreibt (LINK). Alternativ kann die Abbildung XY betrachtet werden, die einen Zeitstrahl mit Werten, sowie die Möglichkeiten die sich durch den Debugger ergeben, aufzeigt.


\subsubsection{Praktische Anwendungsgebiete}
\label{sec:Praktische Anwendungsgebiete}
Elm ist noch recht neu und befindet sich im ständigen Wandel. Für viele Entwickler ist Elm entsprechend noch keine wirkliche Alternative zu ihren gegenwärtig genutzten Frameworks, obgleich die Entwicklung mit den gängigen Werkzeugen oftmals steinig ist. Nur wenige Unternehmen nutzen derzeit Elm in ihrem Produktionsumfeld. Die derzeit größten Nutzer sind NoRedInk, Prezi und CircuitHub. Alle Betriebe überführen Stück für Stück bereits bestehende Teile ihres Frontends zu Elm. NoRedInk gibt an, dass der überführte Elm-Code in den letzten acht Monaten keinerlei Laufzeitfehler erzeugt hat, anders als die vorherige Implementierung \footnote{\cite[Vgl.]{feldman-no-errors}}.
Dennoch wird es wohl noch eine Weile dauern, bis sich mehr Firmen der Vorstellung hingeben ihr lauffähiges System in die vielversprechende Programmiersprache Elm zu portieren, nicht zuletzt, weil sie sich noch in einem sehr frühen Stadium befindet und somit noch nicht völlig ausgereift ist.
Es existiert jedoch bereits eine Vielzahl an Projekten die mit Elm verwirklicht wurden. Dabei sind viele dieser Projekte kleinere Retro-Spiele, die über den Browser gespielt werden können \footnote{\cite[vgl.]{builtwithelm}}.
Dieser Einblick zeigt bereits, dass Elm in vielerlei Hinsicht Besserung für die Entwicklung von Webapplikationen verspricht. Doch wie praktikabel sind diese Versprechungen? Ist die Programmiersprache effizient und intuitiv, oder durch ihr noch frühes Entwicklungsstadium unausgereift?
Auf der offiziellen Webseite von Elm wird angegeben, dass Elm die beste funktionale Programmiersprache des Browsers sei.\footnote{\glqq the best of functional programming in your browser\grqq\cite[Vgl.]{elm-no-runtime-errors}}
Diese Aussage wird im Rahmen dieser Arbeit anhand der Entwicklung, respektive der Überführung einer Webseite in natives Elm überprüft. Die Evaluierung erfolgt schließlich anhand einiger Bewertungskriterien.

\subsection{Einführung in die Elm-Architektur}
\label{sec:elm-architektur}
Vor der Evaluierung der Programmiersprache Elm ist es jedoch notwendig, die Elm-Architektur zu erläutern und einige Grundaspekte zu erklären. Dieses Kapitel soll einen Einstieg in die neue Programmiersprache Elm ermöglichen und stellt nicht nur die Grundkonzepte der Programmiersprache, sondern auch einige mitgelieferte Werkzeuge vor.

\subsubsection{Elm-REPL}
\label{sec:elm-repl}
Ein nützliches Tool um mit Werten in Elm zu interagieren und kleinere Algorithmen zu testen, ist die $elm-repl$. \ac{REPL} bezeichnet dabei die Iteration, welche ein Programmcode durchlebt. Zunächst wird der Quellcode gelesen (read), danach ausgewertet (evaluate) und das Ergebnis ausgegeben (print). Dieser Vorgang wiederholt sich (loop), bis die Entwicklung fertig ist. $Elm-repl$ wird über die Kommandozeile aufgerufen und gestartet. In Abbildung~\ref{fig:elm-repl} ist das Tool abgebildet und zeigt beispielhafte Eingaben und Evaluierungen.
\begin{figure}[h]
\centering
\includegraphics[scale=0.5]{img/elm-repl.png}
\caption{Das Tool elm-repl in der Kommandozeile}\label{fig:elm-repl}
\end{figure}
Wie aus der Abbildung~\ref{fig:elm-repl} ersichtlich wird, werden auch Fehler ausgegeben. In diesem Beispiel wurde versucht auf die Bibliothek $String$, genauer die Funktion $length$ zuzugreifen. Diese Bibliothek wurde allerdings noch nicht eingebunden, wodurch es zu dem Fehler kam. Nachdem die Bibliothek über das Kommando $import\,String$ importiert wurde, war der Fehler behoben. Die $elm-repl$ erlaubt es, alle Bibliotheken die das Paket $core$ mitliefert, zu importieren und Algorithmen zu evaluieren.

\subsubsection{Basisdatentypen}
\label{sec:Basisdatentypen}
Elm hat nur eine geringe Anzahl an Basisdatentypen, mithilfe derer sämtliche weiterführende Konstrukte abgeleitet werden können:
\begin{itemize}
	\item 42 : number
	\item True : Bool
	\item 'a' : Char
	\item {[1, 2, 3] : List}
\end{itemize}
Durch die Kombination dieser Basisdatentypen können weitere komplexere Datentypen wie $Strings$, $Integer$ oder $Floats$, wie es in Abbildung~\ref{fig:elm-repl} zu sehen ist, erzeugt werden.

\subsubsection{Basisfunktionen}
\label{sec:Basisfunktionen}
Elm kommt mit einer Vielzahl an grundlegenden Funktionen, um notwendige arithmetische Operationen zu ermöglichen. Dazu gehören Funktionen zur Addition ($+$), Subtraktion ($-$), Multiplikation ($*$) und Division ($/$). Um Vergleiche zu vollziehen gibt es die Funktionen zur Prüfung der Gleichheit ($==$) und Ungleichheit($/=$). Wie in Abbildung~\ref{fig:elm-repl} zu sehen ist, können zwei $String$s zu einem verbunden werden, indem die Funktion $++$ angewendet wird.

\subsubsection{Union Types}
\label{sec:Union-Types}
In vielen Applikationen können Datentypen unterschiedliche Zustände annehmen. So kann beispielsweise ein Tag die Zustände $Montag$ bis $Sonntag$ annehmen. Die in anderen Programmiersprachen als $algebraische\,Datentypen$ bekannte Aufzählung von Zuständen wird in Elm $Union\,Type$ genannt und beschreibt auch hier eine Aufzählung von endlich vielen Zuständen eines Datentypen. Die Einführung eines solchen $Union\,Type$s erlaubt es dem Compiler den Quellcode auf fehlende Zustände zu überprüfen. Ist es zum Beispiel das Ziel, eine Meldung abhängig vom aktuellen Wochentag auszugeben, müssen alle Zustände und ihre Folge dafür deklariert werden. Fehlt eine Deklaration, kann der $Elm-Compiler$ dies anhand des $Union\,Type$s erkennen.
\begin{figure}[h]
\begin{lstlisting}[language=Elm]
type Day = Monday | Tuesday | .. | Sunday
storeStatus : Day -> String
storeStatus day =
  case day of
    Monday ->
      "Opened"
    Sunday ->
      "Closed"
\end{lstlisting}
\caption{Ein Union-Type in Elm}\label{fig:elm-union-type}
\end{figure}
Betrachtet man die Abbildung~\ref{fig:elm-union-type} ist klar erkennbar, dass die Funktion $storeStatus$ nicht alle Fälle, die der Typ $Day$ annehmen kann, behandelt. Aufgrund der vorherigen Deklaration eines $Union\,Type$ kann der $Elm-Compiler$ einsehen, welche möglichen Zustände der Typ $Day$ annehmen kann. Dadurch wird ein Fehler geworfen, insofern die fehlenden Typen nicht ergänzt werden. Dieses Konzept der algebraischen Datentypen erscheint zunächst sehr ähnlich den $Enumerationen$ in beispielsweise $C++$ oder anderen Programmiersprachen. Tatsächlich gibt es diese Form der Datenabstraktion in einer Vielzahl von funktionalen Programmiersprachen, mit dem wohl bekanntesten Vertreter der funktionalen Programmiersprache $Haskell$. Der Vorteil eines $Union\,Type$s in Elm gegenüber $Enumerationen$ in C++ oder Java ist, dass jeder Zustand der Aufzählung optionale Parameter übergeben bekommen kann. Damit bietet sich die Möglichkeit viel detailliertere Konstrukte zu erzeugen, ohne die Typensicherheit die sich durch die $Union\,Type$s ergibt zu verlieren. Das Beispiel der Abbildung~\ref{fig:elm-union-type} könnte entsprechend erweitert werden, so dass die Tage $Montag$ bis $Freitag$ zusätzlich noch eine Öffnungszeit besitzen.
\begin{figure}[h]
\begin{lstlisting}[language=Elm]
type Day = Monday Int Int | .. | Sunday
storeStatus : Day -> String
storeStatus day =
  case day of
    Monday start end->
      "Opened from:"
      	++ start "until"
      	++ end
    Sunday ->
      "Closed"
\end{lstlisting}
\caption{Ein erweiterter Union-Type in Elm}\label{fig:elm-union-type-advanced}
\end{figure}
Wie in Abbildung~\ref{fig:elm-union-type-advanced} zu sehen ist, wurden die einzelnen Tage in der Deklaration um die Parameter $Int$ erweitert. In diesem Fall sollen die $Integer$ vereinfacht die Start- und Endzeit der Öffnungszeit widerspiegeln. Auffällig ist, dass die Signatur der Funktion $storeStatus$ im Vergleich zur vorherigen Abbildung~\ref{fig:elm-union-type} nicht verändert wurde. Der übergebene Datentyp ist noch immer ein $Day$, mit dem Zusatz, dass die Wochentage zwei weitere Parameter $start$ und $end$ besitzen. Auf diese Weise können komplexe Konstrukte definiert werden, während die Typensicherheit gewahrt bleibt.

\subsubsection{Records}
\label{sec:Records}
Eine weitere Datenstruktur in Elm sind die sogenannten $Records$. Ein $Record$ ist vergleichbar mit einem $Objekt$ in \ac{JS} und verfolgt eine sehr ähnliche Syntax. Es ist möglich, eigene Felder in einem $Record$ zu definieren, mit allen Datentypen die bisher erzeugt wurden. Dazu gehören nicht nur die Basis-Datentypen, sondern auch alle daraus konstruierten. Auch können verschiedene Datentypen in einem $Record$ kombiniert werden.
\begin{figure}[h]
\begin{lstlisting}[language=Elm]
university = { typ = "Fachhochschule"
             , gruendungsdatum = 1969
             , ort = "Kiel" }
university.typ  --> "Fachhochschule"
.ort university --> "Kiel"

{university | ort = "Flensburg" }
\end{lstlisting}
\caption{Ein Record in Elm}\label{fig:elm-record}
\end{figure}
In Zeile $1$ der Abbildung~\ref{fig:elm-record} ist erkennbar, wie ein $Record$ mit initialen Werten erstellt wird. Lediglich das Gleichheitszeichen für die Zuweisung unterscheidet sich hierbei von herkömmlichen erstellen eines $Records$ in \ac{JS}. Um ein Feld in einem $Record$ direkt anzusprechen, kann die übliche Schreibweise aus Zeile $4$ gewählt werden. Zeile $5$ hingegen führt zum gleichen Ergebnis, nutzt intern jedoch eine $anonyme\,Funktion$. Der Unterschied eines $Records$ im Gegensatz zu einem $Objekt$ in \ac{JS} liegt darin, dass Felder nicht dynamisch hinzugefügt oder gelöscht werden können. Sobald ein Feld definiert wurde, ist es für die gesamte Laufzeit vorhanden. Die einzige Möglichkeit einen $Record$ zu verändern liegt darin, den Inhalt der Felder zu manipulieren. Zeile $7$ der Abbildung~\ref{fig:elm-record} zeigt, wie das Feld $Ort$ des $Records$ mit einem neuen Wert versehen wird. Dabei ist zu beachten, dass aufgrund der Unveränderlichkeit einer Datenstruktur in Elm nicht der $Record$ selbst verändert wird. Viel mehr wird ein neuer $Record$ erstellt, der die Änderung, sowie gleichbleibenden Felder des alten $Record$ annimmt. Ein weiterer Unterschied gegenüber \ac{JS} ist, dass ein Feld nie den Wert $undefined$ oder $null$ annehmen kann, da eingeführte Felder stets initialisiert werden müssen. Des Weiteren ist es nicht möglich rekursive Felder zu erzeugen.

\subsubsection{Tupel}
\label{sec:Tupel}
Sollte es notwendig sein, mehr als einen Wert als Rückgabewert zu haben, so bietet sich die Nutzung von $Tupel$n an. Ein $Tupel$ ist eine weitere Datenstruktur in Elm und kann eine beliebig feste Anzahl an Werten beinhalten. Jeder Wert ist unabhängig von den anderen und kann einen geeigneten Datentyp annehmen. Die folgende Abbildung~\ref{fig:elm-tupel} zeigt, wie ein solches Tupel von einer Funktion zurückgegeben werden kann.
\begin{figure}[h]
\begin{lstlisting}[language=Elm]
type Food = { name: String, sort: String }

isCookie : Food -> (Bool, String)
isCookie food =
  if food.sort == "cookie" then
    (True, food.name)
  else
    (False, "No cookie.")

isCookie { name = "Oreo", sort = "cookie" }
--> (True, "Oreo")
\end{lstlisting}
\caption{Ein Tupel in Elm}\label{fig:elm-tupel}
\end{figure}
Der Rückgabewert des beispielhaften Aufrufes in Zeile $10$ ist ein Tupel, bestehend aus einem $Bool$ und einem $String$, wie es in der Signatur in Zeile $3$ definiert wurde. Soll das Tupel noch weiter verwendet werden, liefert die Bibliothek $Basics$ aus dem Paket $elm-lang/core$ die Funktionen $fst$ und $snd$, die entsprechend den ersten oder zweiten Wert eines Tupels zurückliefern. Auf diese Weise kann ein einzelner Wert des Tupels extrahiert werden. Aufgrund der fehlenden Wege Tupel mit mehr als zwei Werten auszuwerten, sollte nur bedingt auf Tupel zurückgegriffen und stattdessen ein $Record$ verwendet werden.

\subsubsection{Variablen}
\label{sec:Variablen}
Die Lebenszeit eines $Records$ ist die gesamte Laufzeit des Programmes. Eine $Variable$ in Elm hingegen bleibt nur für die Dauer der Funktion in der sie definiert wurde erhalten und ist auch nur in diesem Namensraum gültig. Außerhalb der Funktion ist die Variable nicht mehr bekannt. In Elm wird eine Variable innerhalb eines $let..in$-Konstrukts erzeugt. $Let$ bezeichnet dabei den Abschnitt, in welchem sämtliche Variablen erstellt werden und einen Wert zugewiesen bekommen. Die innerhalb dieses Blockes definierten Variablen können dabei aufeinander zugreifen, unabhängig von ihrer Reihenfolge. Mit Hilfe des $let..in$-Blockes kann die Logik einer Funktion noch weiter ausgelagert werden, ohne eine explizit neue Funktion zu definieren.
\begin{figure}[h]
\begin{lstlisting}[language=Elm]
meaningOfLife =
  let
    fourtyTwo = oneHundred - 58
    oneHundred = 100
  in
    fourtyTwo

meaningOfLife --> 42
\end{lstlisting}
\caption{Eine Variablendeklaration in Elm}\label{fig:elm-variables}
\end{figure}
Abbildung~\ref{fig:elm-variables} zeigt beispielhaft die Deklaration einer Funktion $meaningOfLife$, die innerhalb des $let..in$-Blockes die Variablen $fourtyTwo$ und $oneHundred$ deklariert und ihnen einen Wert zuweist. Die Inhalte eines $let..in$-Blockes müssen eingerückt werden, während hingegen eine übliche Funktion auch in einer Zeile verfasst werden kann. Des Weiteren ist erkennbar, dass die Variable $oneHundred$ erst nach der Deklaration von $fourtyTwo$ deklariert wird, jedoch bereits vorher nutzbar ist. Der $elm-compiler$ optimiert diese Programmstelle während des Kompiliervorganges entsprechend. Die beiden deklarierten Variablen sind lediglich in dem definierten $let..in$-Block verfügbar. Eine Verwendung außerhalb des Blockes hätte einen Compilerfehler zur Folge. Es besteht ferner die Möglichkeit ganze Funktionen innerhalb des Blockes zu definieren und zu nutzen.

\subsubsection{Kontrollstrukturen}
\label{sec:Kontrollstrukturen}
Elm bietet die Möglichkeit verschiedene Kontrollstrukturen anzuwenden. Dabei wird ein Record auf mögliche Werte hin überprüft und abhängig vom Wert ein anderer logischer Pfad gewählt.
\begin{figure}[h]
\begin{lstlisting}[language=Elm]
isItACookie food =
  if food == "cookie" then
    True
  else if food == "grapefruit" then
    False
  else
    False
\end{lstlisting}
\caption{Eine Konstrollstruktur in Elm}\label{fig:elm-conditional}
\end{figure}
In Abbildung~\ref{fig:elm-conditional} ist eine herkömmliche $if$-Abfrage zu sehen. In Elm sind die Schlüsselwörter $if$, $then$ und $else$ notwendig und unterteilen die Bedingung von der angewandten Folge, die nach dem Schlüsselwort $then$ folgt. Insofern keine $if$-Bedingung zutrifft, wird der $else$-Fall angewandt.
Elm verfolgt auch hier die statische Typisierung und fordert, dass jede mögliche Verzweigung denselben Typen zurückliefert. Eine Bedingung muss in Elm immer $True$ oder $False$, sprich einen $Bool$ liefern. In anderen Programmiersprachen wie beispielsweise $C++$ evaluiert eine Bedingung zu $True$, wenn sie nicht $0$ oder $false$ ist. Das Beispiel der Abbildung~\ref{fig:cpp-conditional} zeigt, dass $0$ zu $false$ auswertet. In Elm hingegen meldet der $elm-compiler$ einen Fehler, sobald eine Bedingung zu einem anderen Typen als $Bool$ auswertet.
Abbildung~\ref{fig:elm-union-type} zeigt, wie $Union\,Type$s überprüft werden können. Die Kontrollstruktur ähnelt dabei dem $switch$-$case$-Konstrukt von anderen Programmiersprachen wie \ac{JS} oder C++.
\begin{figure}[h]
\begin{lstlisting}[language=cpp]
string whatIsZero() {
  if (0) return "0 equals to TRUE";
  else return "0 equals to FALSE";
}
cout << whatIsZero() << endl;
//=> 0 equals to FALSE

\end{lstlisting}
\caption{Konstrollstruktur in C++}\label{fig:cpp-conditional}
\end{figure}

\subsubsection{Funktionen}
\label{sec:Funktionen}
Um in irgendeiner Art und Weise eine Interaktion mit den erstellten $Records$ zu vollziehen, wird eine Funktion benötigt. Solch ein Programmkonstrukt wird mittels eines Funktionsnamen und einem Gleichheitszeichen realisiert.
\begin{figure}[h]
\begin{lstlisting}[language=Elm]
add : Int -> Int -> Int
add a b =
  a + b

add 3 4  --> 7
\end{lstlisting}
\caption{Eine Funktion in Elm}\label{fig:elm-function}
\end{figure}
Die Abbildung~\ref{fig:elm-function} macht deutlich, wie eine simple Addition zweier Integer-Werte umgesetzt wird. Die Buchstaben $a$ und $b$ deuten an, dass die Funktion $add$ zwei Parameter erwartet. Der gesamte Inhalt nach dem Gleichheitszeichen ist der sogenannte Rumpf einer Funktion und beschreibt den anzuwendenden Algorithmus. Der Funktionsaufruf ist in Zeile 5 sichtbar. Auffällig ist, dass eine Funktion nie explizit einen Wert zurück gibt, wie es in anderen Programmiersprachen wie \ac{JS} oder C++ durch den Befehl $return$ geschieht. Elm gibt implizit die letzte ausführbare Zeile als Rückgabewert zurück. Der beispielhafte Aufruf der Funktion $add$ in Zeile 5 hätte dementsprechend $7$ als Ergebnis. Des Weiteren ist erkennbar, dass Elm ohne die Nutzung von Kommata oder Klammern auskommt. Klammern werden erst notwendig, wenn mehrere Funktionen geschachtelt werden und eine Auswertung der Befehle von links nach rechts nicht ausreicht.
Zusätzlich zu einer explizit benannten Funktion, gibt es die $anonyme$ Funktion. Sie kommt ohne einen Funktionstitel aus und wird an Funktionen höherer Ordnung weitergereicht.
\begin{figure}[h]
\begin{lstlisting}[language=Elm]
List.filter : (a -> Bool) -> List a -> List a
List.filter (\str -> str == "a") ["a", "b", "c", "a"]
--> ["a", "a"]
\end{lstlisting}
\caption{Anwendung einer anonymen Funktion in Elm}\label{fig:elm-anonym-function}
\end{figure}
$List.filter$ ist eine solche Funktion und erwartet eine anonyme Funktion als Parameter, auf Basis derer eine Liste gefiltert wird. Aus der Abbildung~\ref{fig:elm-anonym-function} wird ersichtlich, dass die übergebene Liste, bestehend aus $4$ Einträgen, auf die Präsenz des Buchstabens $a$ hin überprüft wird. Dabei wird jeweils ein Wert an die anonyme Funktion weitergereicht, die in Form der Variable $str$ repräsentiert und mit dem String $a$ auf Gleichheit überprüft wird. Einträge die diesem Vergleich entsprechen, werden einer neuen Liste hinzugefügt. Wird das Ende der zu überprüfenden Liste erreicht, wird die Ergebnisliste als Rückgabewert ausgegeben.

\subsubsection{Signaturen}
\label{sec:Signaturen}
Die jeweils erste Zeile der Abbildung~\ref{fig:elm-function} und \ref{fig:elm-anonym-function} zeigt den Aufbau einer Signatur. Solch eine Signatur beschreibt die Funktion. Dabei sind Informationen über die Anzahl und der jeweilige Typ der Übergabeparameter, sowie der Typ des Rückgabewertes enthalten. Der letzte Typ einer Signatur ist dabei immer der Rückgabewert, die vorherigen Typen stellen die Typen der Übergabeparameter in der übergebenen Reihenfolge dar. Die einzelnen Parameter sind durch einen Pfeil ($->$) voneinander getrennt. Wird eine Funktion anstelle eines einfachen Datentypen als Parameter übergeben, wird dies wie in Abbildung~\ref{fig:elm-anonym-function} durch das Einklammern der Typen angedeutet. Die Typen innerhalb der Klammer stellen wiederum die Typen der Übergabeparameter und Rückgabewerte der Funktion dar.

Anhand der Abbildung~\ref{fig:elm-function} lässt sich erkennen, dass die $add$-Funktion zwei Parameter vom Typ $Int$ erwartet und letzten Endes ein Ergebnis des Typ $Int$ liefert. Fehlt einer Funktion die Signatur, wird der $elm-compiler$ eine Warnung ausgeben und zusätzlich eine passende, jedoch teilweise allgemeinere Signatur ausgeben. Würde die Funktion in Abbildung~\ref{fig:elm-function} keine Signatur enthalten, würde der $elm-compiler$ die Signatur $add : number -> number -> number$ vorschlagen. Da ein $Int$ nur eine Spezifizierung des Basisdatentypen $number$ darstellt, ist das nicht verwunderlich. Der $elm-compiler$ nutzt implizit die eigens erarbeiteten Signaturen bei der Überprüfung des Quellcodes, sollte der Entwickler keine Signatur angegebenen haben.

\subsubsection{Typen Alias}
\label{sec:typ-alias}
Je komplexer ein Record wird, desto länger und unübersichtlicher wird auch die dazugehörige $Signatur$. Dementsprechend bietet sich eine Abkürzung der Signatur an, die in Elm als $type\,alias$ bezeichnet wird. Dabei handelt es sich um eine Repräsentation einer komplexen Datenstruktur in einer kurzen Schreibweise.
\begin{figure}[h]
\begin{lstlisting}[language=Elm]
isOldEnough : { name : String
	      , profile : String
	      , age : Int
	      } -> Bool
isOldEnough user =
	...
\end{lstlisting}
\caption{Definition einer Funktion ohne Typen Alias}\label{fig:no-type-alias}
\end{figure}
In Abbildung~\ref{fig:no-type-alias} wird eine Funktion mit einem komplexen $Record$ als Übergabeparameter beschrieben. An dieser Stelle hat der $Record$ lediglich drei Felder, führt allerdings schon zu einem recht unübersichtlichen Quellcode. Mit Hilfe des $type\,alias$ kann die Signatur gekürzt werden, indem die Repräsentation des $Records$ ausgelagert wird. In Abbildung~\ref{fig:no-type-alias} können die Auswirkung davon betrachtet werden. Nachdem der $type\,alias$ erstellt wurde, kann das Konstrukt mit dem entsprechenden $alias$ angesprochen werden. Zeile $5$ verdeutlicht, wie der $alias$ genutzt wird. Der Algorithmus der Funktion bleibt völlig unberührt, wodurch diese Änderung eher kosmetischer Natur ist.
\begin{figure}[h]
\begin{lstlisting}[language=Elm]
type alias User = { name : String
	          , profile : String
	          , age : Int
	          }
oldEnough : User -> Bool
\end{lstlisting}
\caption{Definition einer Funktion mit Typen Alias}\label{fig:type-alias}
\end{figure}

\subsubsection{Module}
\label{sec:Module}
\begin{figure}[h]
\begin{lstlisting}[language=Elm]
module MyModule exposing(..)
module MyModule exposing (add, anotherMethod)
\end{lstlisting}
\caption{Mögliche Deklarationen eines Elm-Moduls}\label{fig:elm-module}
\end{figure}
Oftmals ist es sinnvoll Quellcode der sich nur auf eine Problemlösung bezieht zu gruppieren. Auch in Elm ist es möglich Code auszulagern in sogenannte $Module$. Sie beschreiben eine Art Container, in dem der Code isoliert von den anderen Projektteilen betrachtet werden kann. Um ein $Modul$ in Elm zu erstellen, muss eine neue Datei vom Typ $elm$ erzeugt werden. Abbildung~\ref{fig:elm-module} zeigt zwei Beispiele, wie eine Quelldatei als Modul deklariert werden kann. Zeile $1$ gibt dabei sämtliche Funktionen die im Modul beinhaltet sind nach außen weiter, sobald das Modul importiert wird. Dies sollte nur gemacht werden, wenn das Modul sich um eine Art Bibliothek handelt, in der sämtliche Funktionen verfügbar gemacht werden sollen. In Zeile $2$ hingegen wird ersichtlich, wie nur ausgewählte Funktionen nach außen sichtbar gemacht werden. Auf der anderen Seite hingegen würde der Entwickler das Modul $MyModule$ importieren und daraufhin die Funktionen $add$ und $anotherMethod$ nutzen können.

\subsubsection{Importierung}
\label{sec:Importierung}
Jeder Entwickler hat ferner die Möglichkeit die zuvor erstellten Module an einer anderen Stelle zu importieren. Dabei wird das gesamte Modul in den aktuellen Namensraum geladen und nutzbar gemacht. In welcher Ausprägung die Funktionen des importierten Moduls eingebunden werden, hängt von der Spezifikation des Imports ab.
\begin{figure}[h]
\begin{lstlisting}[language=Elm]
import MyModule
import MyModule exposing(..)
import MyModule exposing (add, anotherMethod)
import MyModule as MyVeryOwnModuleName
\end{lstlisting}
\caption{Mögliche Formen der Importierung eines Elm-Moduls}\label{fig:elm-import-module}
\end{figure}
Die Abbildung~\ref{fig:elm-import-module} zeigt vier mögliche Arten das Modul $MyModule$ zu importieren. Zunächst einmal werden alle Funktionen, die das Modul selbst über das Stichwort $exposing$ freigibt geladen. Das Einbinden wie in Zeile $1$ hat zur Folge, dass sämtliche Funktionen mit dem expliziten Modulnamen voran angesprochen werden müssen. Die Funktion $add$ wird beispielsweise mit $MyModule.add$ aufgerufen. Zeile $2$ wiederum hat zur Folge, dass alle Funktionen des Moduls $MyModule$, darunter auch die Funktion $add$, in den Namensraum des aktuellen Moduls geladen werden. Logischerweise ist an dieser Stelle für den Aufruf der $add$-Funktion nichts weiter notwendig, außer wenn das aktuelle Modul eine gleichnamige Funktion besitzt. In diesem Fall ist die Nutzung analog der Anwendung aus dem vorherigen Beispiel. Die Einbindung in Zeile $3$ wirkt sich reduzierender auf den aktuellen Namensraum aus. Damit ist gemeint, dass nur die explizit genannten Funktionen nach dem Stichwort $exposing$ in den Namensraum geladen werden, alle anderen Funktionen müssen mit dem entsprechenden Modulnamen vorweg angesprochen werden. Schlussendlich kann der Namensraum des eingebundenen Moduls durch den Zusatz des $as$ Stichwortes, gefolgt vom gewünschten Namensraum verändert werden. Die Nutzung der Funktionen ist analog zu den vorherigen Beispielen und kann beliebig kombiniert werden.


\subsubsection{Online-IDE}
\label{sec:Online-IDE}
Damit begonnen werden kann mit Elm zu programmieren, ist es nicht notwendig sämtliche Tools auf dem lokalen Gerät zu installieren. Vielmehr haben die Entwickler von Elm eine Online-\ac{IDE} erstellt, mit der sofort online entwickelt werden kann.
\begin{figure}[h]
	\centering
	\includegraphics[scale=0.4]{img/elm-try.png}
	\caption{Die Online-\ac{IDE} von Elm}\label{fig:elm-try}
\end{figure}
Die Abbildung~\ref{fig:elm-try} zeigt diese \ac{IDE} mit einem typischen $Hello-World$ Beispiel. Auf der linken Seite ist eigentliche \ac{IDE} erkennbar, während die rechte Seite das Ergebnis nach dem Kompiliervorgang ausgibt. Sollte es einen Fehler während des kompilierens geben, wird die entsprechende Fehlermeldung des $elm-compiler$ dort ausgegeben. Mit Hilfe der Schaltfläche $Lights$ können zum Einen die Farben der \ac{IDE} invertiert werden, während der Knopf $Compile$ den geschriebenen Quellcode kompiliert und ausführt. Kleinere Projekte können entsprechend bequem mit diesem Tool realisiert werden. Die \ac{IDE} bietet einen schnellen Einstieg in die Programmiersprache.


\subsubsection{Installation}
\label{sec:Installation}
Die Online-\ac{IDE} bietet einen schnellen Einstieg in die Programmiersprache, ohne Programme lokal installieren zu müssen. Jedoch sind manche Konzepte nicht in der Online-\ac{IDE} nutzbar, so dass Elm lokal installiert werden muss. Das ist beispielsweise der Fall, sobald spezifische Pakete aus des Paketmanager von Elm installiert und benutzt werden möchten. Mehr Informationen über den Elm-eigenen Paketmanager folgen in der Sektion~\nameref{sec:start-eines-projektes}.

Die Elm-Webseite liefert Anleitungen um Elm auf Windows, Mac und allen NodeJs-kompatiblen Betriebssystemen zu installieren. Im folgenden werden die Kommandos zur Installation von Elm und dem \ac{NPM} unter Ubuntu 14.04 64bit erläutert, da der praktische Teil dieser wissenschaftlichen Arbeit unter diesem Zielsystem vorgenommen wurde.
Die Programmiersprache Elm wird über den \ac{NPM} verbreitet und muss somit zuvor installiert werden. Da der \ac{NPM} wiederum auf der Platform NodeJs basiert, muss auch das dazugehörige Paket installiert werden. Für die Installation sind die Kommandos aus Abbildung~\ref{fig:npm-install} in der Kommandozeile auszuführen.
\begin{figure}[h]
\begin{lstlisting}[language=bash]
$ sudo apt-get update
$ sudo apt-get install nodejs
$ sudo apt-get install npm
\end{lstlisting}
\caption{Installation des \ac{NPM}}\label{fig:npm-install}
\end{figure}
Dadurch wird das NodeJs-Paket installiert und ausführbar gemacht. Anschließend sollte überprüft werden, ob die Installation erfolgreich war und sowohl NodeJS, als auch der \ac{NPM} verfügbar sind.
\begin{figure}[h]
\begin{lstlisting}[language=bash]
$ node -v && npm -v
$ npm install -g elm
\end{lstlisting}
\caption{Installation von Elm}\label{fig:npm-install-check}
\end{figure}
Das Kommando in Zeile $1$ der Abbildung~\ref{fig:npm-install-check} liefert ein Ergebnis, wie es in Abbildung~\ref{fig:npm-installed} zu sehen ist. Elm kann nun über den \ac{NPM} mit dem Kommando aus Zeile $2$ installiert werden. Das Kürzel $-g$ installiert die neueste Elm-Version global für alle Projekte auf dem System. Entfernt man es, wird die Zielversion nur für den aktuellen Ordner zugänglich gemacht. Um die in dieser wissenschaftlichen Arbeit behandelte Version $0.17$ zu installieren, ist der Zusatz $@0.17$ direkt hinter dem globalen Flag notwendig. Nachdem alle Kommandos fehlerfrei ausgeführt wurden sollte Elm über die Konsole nutzbar sein.
\begin{figure}[hb]
\centering
\includegraphics[scale=0.3]{img/npm-nodejs-installation.png}
\caption{Überprüfung der erfolgreichen Installation von NodeJs und dem \ac{NPM}}\label{fig:npm-installed}
\end{figure}


\subsubsection{Start eines Projektes}
\label{sec:start-eines-projektes}
Nachdem die Installation von Elm erfolgreich war, kann nun ein neues Projekt lokal erstellt werden. Dazu gehört zunächst einmal die Initialisierung des Projektordners, wodurch notwendige Ordner erstellt und Pakete installiert werden. Die Entwickler von Elm stellen dafür den Paketmanager $elm-package$ bereit. Dieses Tool wird zur Installation von externen Modulen, sogenannte Bibliotheken oder Pakete, benutzt. Zusätzlich hilft der Paketmanager dabei, das eigene Projekt lauffähig zu halten und nicht durch auftretende Aktualisierungen einzelner Pakete die Lauffähigkeit des eigenen Projektes zu verletzen. Dabei verfolgen die Entwickler des Paketmanagers bestimmte Versionsregeln, die auf jedwede Änderung an einem Paket angewendet wird. Alle Versionsnummern haben dabei genau drei Felder $MAJOR.MINOR.PATCH$, wobei der Beginn der Versionsnummern stets $1.0.0$ ist. Die Felder der Versionsnummern ändern sich abhängig der Änderungen an der API eines Paketes. Die harmloseste Änderung ist dabei der $PATCH$. Das bedeutet, dass die \ac{API} unverändert ist und keinerlei Gefahr für die Lauffähigkeit besteht. Es kann sich hierbei um beispielsweise Verbesserungen eines Algorithmus oder andere interne Änderungen handeln, die jedoch für den außenstehenden Entwickler keine Bedeutung haben.
Der nächstgrößere Schritt stellt die Aktualisierung der $MINOR$ Versionsnummer dar. Wie bei der Aktualisierung auf die nächste $PATCH$-Version, ist auch hier eine Aktualisierung unbedenklich. Das $MINOR$-Feld sagt aus, dass neue Methoden hinzugefügt wurden, alte Methoden jedoch unberührt blieben.
Wichtige Änderungen an einem Paket bei der alte Methoden maßgeblich verändert, oder sogar gelöscht wurden, werden mit dem Feld $MAJOR$ ausgedrückt. Hat sich dieses Feld verändert, wird eine Aktualisierung aller Wahrscheinlichkeit nach dazu führen, dass ein Programm nicht mehr lauffähig ist und Änderungen am eigenen Quellcode vorgenommen werden müssen.
Nutzt ein Entwickler beispielsweise das Paket $elm-lang/html$ mit der Version $1.2.3$, wird der importierende Programmcode voraussichtlich kompatibel mit allen zukünftigen Versionen des Pakets bis Version $2.0.0$ sein. Ab diesem Zeitpunkt sind die vorgenommenen Änderungen so massiv, dass eine Aktualisierung nicht ohne weiteres möglich ist. Der Paketmanager $elm-package$ erzwingt die beschriebene Versionierung bei jedem Paket, bevor es veröffentlicht werden kann.\cite[Vgl.]{semantic-versioning}.
\begin{figure}[h]
\centering
\includegraphics[scale=0.3]{img/elm-setup.png}
\caption{Installation eines externen Paketes über den Elm-Paketmanager}\label{fig:elm-install-package}
\end{figure}
Die Abbildung~\ref{fig:elm-install-package} zeigt, wie ein Paket installiert und der Projektordner eingerichtet wird. Der Paketmanager installiert nach der Bestätigung das angeforderte Paket und alle darin enthaltenen, externen Abhängigkeiten. Des Weiteren wird die Datei $elm-package.json$ erstellt, mit grundlegenden Informationen über das eigene Projekt, wie beispielsweise die Versionsnummer, eine Zusammenfassung oder die zu verwendende Lizenz, sowie eine Liste von installierten externen Paketen. Der Ordner $elm-stuff$ enthält die zuvor installierten Pakete. Nun muss noch manuell eine $.elm$-Datei erstellt werden, üblicherweise mit dem Namen des Projektes. In dieser Datei können die zuvor gezeigten Konzepte genutzt werden, um das gewünschte Programm zu verwirklichen.


\subsubsection{Fertigstellung}
\label{sec:elm-compile}
Zum Kompilieren eines Elm-Programmes wird das Tool $elm-make$ genutzt. Es kompiliert den gesamten Quellcode eines Ordners, oder die Dateien, welche dem Befehl hinzugefügt wurden. Ferner kann der Entwickler entscheiden, ob eine $.html$-Datei erstellt werden soll, in der bereits das kompilierte Elm-Programm eingebaut wird, oder eine $.js$-Datei, so dass das Einbinden oder die Auslieferung selbstständig vorgenommen werden kann.

\subsubsection{Elm Reactor}
\label{sec:elm-reactor}
\begin{figure}[h]
\centering
\includegraphics[scale=0.5]{img/elm-reactor.png}
\caption{Der gestartete Elm-Webserver}\label{fig:elm-reactor}
\end{figure}
Wahlweise kann während der Entwicklung das mitgelieferte Tool $elm-reactor$ zum Testen der Elm-Applikation genutzt werden. Der $elm-reactor$ stellt eine Art Webserver dar, durch den die Applikation automatisch bei jedem Aufruf kompiliert und gestartet wird, um sie dem Nutzer zugänglich zu machen. Der Nutzer kann die Applikation im Browser über das lokale Netzwerk öffnen und mit ihr interagieren. Die Abbildung~~\ref{fig:elm-reactor} zeigt den gestarteten Webserver in der Kommandozeile. Er wird standardmäßig auf dem Port $8000$ gestartet.


\chapter{Evaluierung der Programmiersprache Elm}
\label{sec:Evaluierung der Programmiersprache Elm}
Die Programmiersprache Elm wird mit Hilfe einer empirischen Analyse in Form der Entwicklung eines praktischen Beispieles evaluiert. Dabei wird ein fertiges Template einer \ac{SPA}, welches in \ac{HTML}, \ac{CSS} und \ac{JS} programmiert wurde, in nativen Elm-Code überführt, insofern möglich. Ausschließlich die vorhandenen \ac{JS} und \ac{CSS}-Dateien sollen bestehen bleiben. Zur aussagekräftigen Evaluierung der Programmiersprache Elm hinsichtlich einer Nutzung im Bereich der Webentwicklung bedarf es zunächst einiger Bewertungskriterien, anhand derer eine Aussage möglich ist. Nachdem ein Bewertungsmuster erstellt und erläutert wurde, befasst sich der darauf folgende Teil der wissenschaftlichen Arbeit mit der Überführung des Templates in nativen Elm-Code. Zuletzt werden Beobachtungen, die während der Entwicklung auftraten, erläutert und die zuvor erörterten Kriterien ausgewertet. Das Kapitel mündet schließlich in einem Fazit, in dem sämtliche Beobachtungen, die Auswertung anhand der Bewertungskriterien abschließend wertend zusammengefasst werden.

\section{Bewertungsmuster}
\label{sec:Bewertungsmuster}
Das fertige Template ist eine \ac{SPA} mit Elementen, wie sie typischerweise auf einer solchen Webseite vertreten sind. Eine \ac{SPA} ist, wie der Name bereits suggeriert, eine Webseite mit effektiv nur einer aktiven Seite und ohne Unterseiten (ausgenommen Impressum, AGB und Datenschutz). Die \ac{SPA} wird genutzt um ein Produkt oder Konzept schnell und einfach zu präsentieren, ohne den Nutzer mit Informationen zu überfluten und Unübersichtlichkeit in Form von tief verlinkten Unterseiten zu erzeugen. Oftmals wird eine SPA auch als Startseite benutzt und bietet nur eine geringe Anzahl an Funktionen.
Die fertige \ac{SPA} soll unter anderem die folgenden, typischen Elemente enthalten:
\begin{itemize}
	\item Navigation mit Anchor-Elementen
	\subitem{-} Verkleinern der Navigation nach $x$ Pixeln
	\subitem{-} ScrollSpy zur Darstellung der aktuellen Position
	\item Titelbild mit einem vertikal und horizontal zentriertem Text
	\item Service-Sektion
	\item Twitter-Bootstrap-\ac{CSS} mitsamt allen Funktionen
	\subitem{-} Vorgefertiges, responsive Design
	\subitem{-} Aufklappbares Menü
	\item Portfolio mit Bildern, wobei ein Klick einen asynchronen Request ausführt und Daten nach lädt
	\item Formular zur Kontaktaufnahme mit Validierung der eingegebenen Daten
\end{itemize}
Mit diesen Elementen kann eine typische SPA verwirklicht werden. Die Navigation bietet dabei die Möglichkeit für den Nutzer schnell zwischen einzelnen Sektionen der Seite zu wechseln. Das initiale Titelbild mit einem zentrierten Text gibt den Kontext der Präsentation an und soll das Interesse des Nutzers anregen. Die folgende Service-Sektion wird dazu genutzt, allgemeine Informationen über das beworbene Produkt anzuzeigen. In der darauf folgenden Portfolio-Sektion werden dem Nutzer mehrere Bilder des Produktes angezeigt, wobei ein Klick auf eines der Bilder dazu führt, dass ein Popup erzeugt wird, in welches mit Hilfe eines AJAX-Requests Informationen asynchron vom Server angefordert und im Nachhinein geladen werden. Zuletzt kann sich der Nutzer ein Kontaktformular ausfüllen, das auf Korrektheit hin überprüft wird.


\section{Allgemeine Bewertungskriterien}
\label{sec:Bewertungskriterien}
Während der Überführung des Templates soll der erzeugte Code, sowie der Weg dahin analysiert werden. Hierbei sind Aspekte wie Wiederverwendbarkeit und Effizienz von großer Bedeutung. Aber auch die Produktivität während der Arbeit mit dem Code soll betrachtet werden. Die Bewertungskriterien setzen sich zum Großteil aus den zugrunde liegenden Kriterien aus dem Dokument XY der Washington University zusammen. Im Folgenden soll die Notwendigkeit der Kriterien und ihre eigentliche Bedeutung verständlich gemacht werden.

%TODO: http://www.cs.gordon.edu/courses/cs323/lectures-2009/LanguageEvaluationCriteria.pdf\\
%http://courses.cs.washington.edu/courses/cse341/02sp/concepts/evaluating-languages.html

%\subsection{Entwicklungsgeschwindigkeit}
%\label{sec:Entwicklungsgeschwindigkeit}
%Gerade für Startups ist es wichtig, ein erstes Produkt so schnell wie möglich bereitzustellen. Die Entwickler müssen entsprechend mit wenigen Mitteln ein fertiges (Software-)Produkt ausliefern können, selbst wenn sie noch kein oder wenig Vorwissen zu einer Programmiersprache haben.
%Dementsprechend sollte eine Programmiersprache nur eine geringe Anzahl an primitiven Konzepten aufweisen, die sich leicht erweitern lassen. Beispielsweise hat die Programmiersprache $C$ standardmäßig keine Anreihung von Buchstaben, auch bekannt als $String$. Jedoch gibt es den Datentyp $char$, der einen einzelnen Buchstaben repräsentiert. Durch die Verknüpfung mehrerer $char$'s, kann letztendlich der Datentyp $String$ umgesetzt werden. 



\subsection{Wartbarkeit und Lesbarkeit}
\label{sec:muster_wartbarkeit_und_lesbarkeit}
Es ist unabdingbar, dass verfasster Quellcode wartbar ist. Dazu gehört einerseits, dass der Code lesbar ist, unabhängig von der Zeit die sich ein Entwickler bereits mit der Codebasis auseinandergesetzt hat. Damit Quellcode lesbarer wird, reicht es schon aus Kommentare im Code zuzulassen, die beispielsweise eine Funktion und ihr Ziel beschreiben, oder wichtige Informationen über einen Algorithmus enthalten. Des Weiteren sollten Funktionen und Variablen kurze und prägnante Namen haben, wodurch die Verständlichkeit des Quellcodes unterstützt wird. Die Programmiersprache Elm kann als wartbar bezeichnet werden, insofern es Möglichkeiten zum Kommentieren des Codes gibt, oder bestenfalls automatisch Informationen in Form von Kommentaren über Funktionen und Verhaltensweisen von Algorithmen generiert werden.


\subsection{Zuverlässigkeit}
\label{sec:muster_zuverlaessigkeit}
Für einen Entwickler von Software ist es wichtig, dass das ausgelieferte Programm letzten Endes fehlerfrei funktioniert. Dazu gehört, dass das Programm nicht unvorhergesehen abstürzt, andere Systeme beeinträchtigt, oder dem späteren Nutzer der Software anderweitig Probleme beschert. In den meisten Fällen helfen die Editoren mit denen der Quellcode geschrieben wird bereits, indem syntaktische Fehler durch Markierungen sichtbar, oder Vorschläge zur Vervollständigung des angefangenen Codes gemacht werden. Wichtig ist entsprechend, dass die Programmiersprache auf offensichtliche Fehler, entweder durch Plugins für den Editor oder durch den Compiler selbst, aufmerksam macht und sie somit verhindert und nicht erst im Produktionssystem den Fehler zulässt. Als Testfall wird Quellcode bewusst mit Fehlern versehen, die zu einem Absturz des Programmes während der Laufzeit, oder zu anderen Problemen führen würden. Erkennt der $elm-compiler$ diese Fehler oder umgeht den Absturz, gilt das Kriterium als erfüllt.


\subsection{Portabilität}
\label{sec:muster_portabilitaet}
Häufig sind Programmiersprachen nicht auf allen Systemen lauffähig, um so Applikationen zu erstellen. Es kommt dabei sehr stark auf die Hardwarekomponenten und das Betriebssystem des Zielsystems an. Es ist wünschenswert, dass Quellcode nur einmal geschrieben und auf das Zielsystem übertragen werden kann, ohne großartige Änderungen am Quellcode vornehmen zu müssen. Letzten Endes wird versucht den Quellcode auf mehreren Zielsystemen zu kompilieren. Elm liefert Installationsanweisungen für die Betriebssysteme $Mac$, $Windows$ und allen Betriebssystemen, die $nodejs$ unterstützen. Gibt es keinerlei Differenz in Form von Fehlern oder Warnungen während des Kompilierens, gilt die Portabilität als erfüllt.


\subsection{Effizienz}
\label{sec:muster_effizienz}
„Zeit ist Geld.“ ist auch heute noch immer eine wahre Aussage. Entsprechend ist es von Vorteil, wenn Quellcode schnell erzeugt, getestet und als fertiges Produkt (Software) an den Kunden ausgeliefert werden kann. Dabei ist es wichtig, dass der Compiler den Quellcode schnell in ein lauffähiges Programm verwandelt und auch das daraus resultierende Programm effizient arbeitet, sprich schnell ist. Damit eine Aussage über die notwendige Zeit für das kompilieren des Quellcodes getroffen werden kann, wird der Code zehn mal kompiliert und die benötigte Zeit gemessen. Anschließend wird der höchste und niedrigste Wert verworfen. Um ein Caching durch den Compiler zu verhindern, werden vor jedem Kompiliervorgang die zuvor erzeugten Programmdateien im Ordner $elm-stuff$ gelöscht. Im Anschluss wird das arithmetische Mittel der verbleibenden acht Werte ermittelt.
Die Performance der Programmiersprache wird nicht anhand des hier entwickelten Projektes, sondern der externen Applikation \cite{https://github.com/evancz/todomvc-perf-comparison/} TodoMVC Performance Comparison ermittelt. Dieses Tool ermöglicht es, anhand der Beispiel-Applikation $TodoMVC$, die in mehreren Programmiersprachen realisiert wurde, die Performance der Applikation in den einzelnen Programmiersprachen zu testen. Dabei werden automatisch gewisse Aktionen in der jeweiligen Sprache durchgeführt, die benötigte Zeit gemessen und die Ergebnisse am Ende gegenübergestellt. Mit Hilfe des Tools werden 20 Messungen durchgeführt und das Endergebnis betrachtet. Ist Elm signifikant schneller als der Großteil der anderen getesteten Programmiersprachen, gilt das Kriterium als erfüllt. Die anderen Applikationsumgebungen sind ebenso in nativem \ac{JS} oder einer zu \ac{JS} kompilierenden Programmiersprache geschrieben. Dadurch kann eine Aussage über die tatsächliche Effizienz der Applikation im Vergleich gemacht werden.


\subsection{Wiederverwendbarkeit}
\label{sec:muster_wiederverwendbarkeit}
Vorhandene Codeteile neu zu verfassen oder händisch kopieren zu müssen ist sehr ineffizient. Besser ist es, wenn Funktionen mehrfach genutzt werden können. Auf diese Weise müssen Änderungen nicht mehrmals vorgenommen werden und die Fehleranfälligkeit sinkt. Ferner sollten Funktionen nicht nur wiederverwendbar, sondern isoliert in einem eigenen Modul definiert werden können. Besteht die Möglichkeit Module zu erstellen und diese an unterschiedlichen Stellen beliebig anzuwenden, gilt das Kriterium als erfüllt. Erweitert wird das Kriterium durch einen zusätzlichen Zugriffsschutz von außen, das bedeutet, dass innerhalb eines Moduls definiert werden kann, welche Funktionen nach außen hin sichtbar sind.


\section{Web-spezifische Bewertungskriterien}
\label{sec:muster_web}
Die bisherigen Kriterien ermöglichen es, die Programmiersprache als solches anhand der eingebauten Möglichkeiten die diese bietet objektiv beurteilen zu können. Da sich diese wissenschaftliche Arbeit allerdings ausdrücklich mit der Evaluierung von Elm für Webapplikationen auseinandersetzt, müssen auch diese Kriterien untersucht werden und einen höheren Stellenwert bei der Bewertung einnehmen.
Webapplikationen können grundsätzlich in zwei Typen unterteilt werden. Zunächst gibt es die \ac{SPA}, bestehend aus nur einer einzigen Seite, bei der typischerweise nur wenig  Programmlogik vorhanden ist und der Client lediglich die Informationen anzeigt. Des Weiteren gibt es die Rich-Internet-Applikationen\\
(RIA: http://www.itwissen.info/definition/lexikon/rich-Internet-application-RIA.html). Diese Art der Webapplikation besitzt im wesentlichen mehr Programmlogik, die der Client ausführen kann. Ein weiteres Merkmal sind oftmals auch die Anzahl der vorhandenen Unterseiten, wodurch wiederum mehr Logik erforderlich wird.
Bei der Erstellung einer \ac{SPA} muss generell weniger Aufwand betrieben werden, um eine fertige Präsentation zu erstellen.
Jede Webapplikation lebt von einem Frontend, sowie einem Backend. Typischerweise bezeichnet das Frontend dabei alle Komponenten, die an den Benutzeroberfläche des Nutzers gesendet werden. Dazu gehören unter anderem die Komponenten \ac{HTML}, \ac{CSS} und \ac{JS}. Der Nutzer interagiert mit dieser Darstellung der Oberfläche.
Das Gegenstück zum Frontend stellt das sogenannte Backend dar. Dabei handelt es sich um Komponenten, die dem Webserver zugehörig sind. Unter anderem ist das zum Beispiel eine Datenbank, die eigentliche Webapplikation und natürlich der Webserver selbst. Der Nutzer agiert mit dem Backend nur über zuvor im Frontend eingerichtete Schnittstellen.


\subsection{Browser Kompatibilität}
\label{sec:muster_browser_kompatibilitaet}
Ein weiterer wichtiger Aspekt ist die Kompatibilität der Browser mit den genutzten Programmiersprachen. Da der Browser die Darstellung des \ac{HTML} und \ac{CSS} Quellcodes übernimmt, sowie die Manipulationen des DOM durch \ac{JS}, ist es wichtig, dass der Browser den vorhandenen Quellcode lesen und ausführen kann. Sämtliche Sprachen wie \ac{HTML}, \ac{CSS} und \ac{JS} befinden sich im konstanten Wandel und werden stets weiter entwickelt.
Dabei werden nicht nur vorhandene Fehler behoben, sondern auch neue Eigenschaften hinzugefügt, sowie teilweise ersetzt oder verworfen. Diese Änderungen können darin münden, dass Nutzer unterschiedlicher Browser, auch unterschiedliche Ergebnisse angezeigt bekommen, manche Features gar nicht erst funktionieren oder die Applikation im schlimmsten Fall abstürzt. Dementsprechend ist es wichtig, dass die Applikation auf den gängigen Browsern fehlerfrei funktioniert, insbesondere den fünf meistgenutzten Browsern Google Chrome, Safari, Internet Explorer, Firefox und Opera (vgl. Abbildung~\ref{fig:browser-may-2016}). Sollte die Applikation fehlerfrei in allen Browsern starten und die Funktionalität vollends gegeben sein, kann dieses Kriterium als erfüllt angesehen werden.
\iffalse
www.w3counter.com/trends\\+
https://www.w3counter.com/globalstats.php?year=2016&month=5
\fi

\begin{figure}[hb]
  \centering  
  \includegraphics[scale=0.3]{img/browser_2016.png}
  \caption{Die fünf meistgenutzten Browser im Mai 2016 \cite{top-five-browser-statistics}}\label{fig:browser-may-2016}
\end{figure}

\subsection{Interoperabilität}
\label{sec:muster_interoperabilitaet}
Auch bei den Webapplikationen ist es wichtig, bereits existente Frameworks und Problemlösungen nutzen zu können. Folglich müssen externe \ac{JS} und \ac{CSS} Bibliotheken ohne große Probleme eingebettet werden können, ohne den Quellcode zu verändern. Sollte es die Möglichkeit geben bestehende, externe Bibliotheken einzubinden oder anderweitig nutzbar zu machen, gilt das Kriterium als erfüllt.


\subsection{Asynchrone Verarbeitung}
\label{sec:muster_asynchrone_verarbeitung}
Große Datenmengen, deren Übertragung einige Sekunden in Anspruch nimmt, oder Aufgaben die langwierig sind, sollten nicht synchron ausgeführt werden. Stattdessen bietet es sich an, asynchrone Aufgabenblöcke zu definieren, wie beispielsweise das nachladen von Daten, sollten sie tatsächlich angefordert werden. Am Beispiel der hier geplanten \ac{SPA} bedeutet dies, dass Informationen von einem Webserver erst angefordert und sichtbar gemacht werden, sobald der Nutzer diese durch einen Klick auf ein Element anfordert. Da initial gewisse Daten noch nicht vorhanden sind, verringert sich die Datenmenge der zu übertragenen Informationen. Dadurch kann der Webserver entlastet und die initiale Ladezeit der Webseite verringert werden. Um das Kriterium der asynchronen Verarbeitung zu erfüllen, sollte Elm die Möglichkeit liefern, asynchrone Requests an einen Server zu schicken, Daten anzufordern und die Antwort darzustellen, oder andere langwierige Aufgaben asynchron zu verarbeiten. Die Applikation darf dabei nicht in einen undefinierten Zustand kommen, sprich durch beispielsweise eine langsame Übertragungsrate Fehler bei der Darstellung erzeugen oder die Ausführung von nachfolgenden Operationen verzögern.


\subsection{Dateigröße}
\label{sec:muster_dateigroesse}
Die Größe einer Datei wirkt sich auf die Dauer der Übertragungszeit aus. Ist die Datenmenge groß, steigt entsprechend auch die Dateigröße an. Eine schlechte Übertragungsrate kann somit zu erheblichen Verzögerungen der fertigen Darstellung führen. Um die Dateigröße von üblichen \ac{JS}-Dateien zu verringern, werden oftmals externe Tools zur Entfernung von Leerzeichen oder der Verkürzung von Funktions- und Variablennamen genutzt. Für die anschließende Bewertung dieses Kriteriums, sollte der Compiler die Möglichkeit zur automatischen Verkleinerung der Dateigröße mit einer Effizienz von mindestens 50\% bieten. Zusätzlich sollte die Dateigröße eines Elm-Programmes mit den eingebundenen Standard-Bibliotheken $elm-lang/core$ und $elm-lang/html$ die Dateigröße der Frameworks $AngularJs$, $Ember$ und $React$ nicht übersteigen. Die Größen der einzelnen Frameworks werden dabei der Statistik von Anton Vynogradenko, verfügbar unter \url{https://gist.github.com/Restuta/cda69e50a853aa64912d} entnommen. Dabei wird jeweils die neueste Version der Statistik gewählt. Dem zugeordnet werden die Ergebnisse für Elm.

\newpage
\section{Empirische Analyse}
\label{sec:Empirische Analyse}
Die folgende Sektion unterstützt und erläutert die praktische Ausarbeitung der \ac{SPA}. Dabei wird zunächst der typische Programmablauf der Applikation erörtert. Ferner wird die Entwicklungsumgebung unter der die Applikation ausgearbeitet wurde vorgestellt. Die Entwicklung wird in einzelne Abschnitte unterteilt und einzeln erläutert.

\subsection{Programmablauf}
\label{sec:Programmablauf}
Abbildung XY: Zeigt die Kommunikation zwischen Client und Webserver
Die Abbildung XY zeigt die geplante Kommunikation zwischen dem Client und dem Webserver.
In Schritt 1 fordert der Nutzer die Webseite an und erzeugt dadurch einen Request. Dieser geht in Schritt 2 bei dem Webserver ein und wird verarbeitet. Abhängig von der angeforderten URL erzeugt der Webserver eine Antwort, mit allen zu sendenden Informationen und dem dazugehörigen \ac{HTML}, \ac{CSS} und \ac{JS} Code. Im nächsten Schritt 3 werden diese Daten zurück an den Client gesendet. Der Client nimmt die Daten entgegen, wie in Schritt 4 beschrieben. Des Weiteren macht der Client die Daten entsprechend sichtbar, so dass der Nutzer eine vollständige Präsentation der Webseite sieht. Schritt 5 beschreibt die mögliche Interaktion des Nutzers mit dem ihm präsentierten Dokument. Jede Interaktion wird durch \ac{JS} erkannt und erzeugt eine Reaktion. Unter Umständen ist dies ein Request an den Server, womit bei Schritt 1 begonnen wird. Externe Daten wie Stylesheets oder bestehende \ac{JS}-Skripte  werden asynchron nachgeladen, wie es Schritt 6 andeutet. Dabei kann dieser Schritt zeitlich gesehen zu jedem Zeitpunkt nach Schritt 3 geschehen.
%TODO: Abbildung einfügen: Seite 68


\subsection{Entwicklungsumgebung}
\label{sec:Entwicklungsumgebung}
Die zugrunde liegende Elm-Version ist 0.17 und wird unter Ubuntu 14.04 64bit installiert. Als Editor wird Atom mit den Elm-spezifischen Plugins $language-elm$, $linter-elm-make$ und $elm-format$ verwendet. Diese Plugins unterstützen die Entwicklung von Elm-Programmen, indem der Quellcode beim speichern automatisch dem Style-Guide entsprechend formatiert, eventuelle Fehler bei der Kompilierung direkt im Editor sichtbar gemacht und der Code syntaktisch gefärbt, sowie Vorschläge zur Vervollständigung des geschriebenen Quellcodes gemacht werden. Da die Pakete ständig aktualisiert und verändert werden, wird an dieser Stelle von einer detaillierten Beschreibung zur Installation und Verwendung abgesehen und auf die Dokumentationen der einzelnen Pakete verwiesen.
Der angefertigte Quellcode wird durch den Compiler $elm-make$ kompiliert.
Anschließend wird die fertige \ac{SPA} mit den Browsern Google Chrome (Version 51), Internet Explorer (Version XY), Mozilla Firefox (Version 47) und Opera (Version 38). In Bezug auf die Abbildung~\ref{fig:browser-may-2016} fällt auf, dass Safari als einziger der fünf meistgenutzten Browser nicht getestet wird. Zum Zeitpunkt dieser wissenschaftlichen Arbeit stand kein entsprechendes Testgerät zur Verfügung. Die in der Abbildung genannten Browser decken 94,9\% aller genutzten Browser ab. Abzüglich des Browsers Safari, werden entsprechend die verbleibenden 82,2\% aller Browser überprüft.


TODO: (https://github.com/triforkse/atom-elm-format)\\
(https://github.com/triforkse/atom-elm-format)\\
(https://atom.io/packages/language-elm\\

\subsection{Grundaufbau}
\label{sec:Grundaufbau}
Damit der Nutzer letzten Endes eine Webseite besuchen kann, muss zunächst ein auslieferbares Dokument erzeugt werden. Hierfür wird zunächst die $index.html$-Datei aus dem Template $Agency$ übernommen und entsprechend angepasst. Die Datei dient dabei als Grundgerüst für die Elm-Applikation und lädt externe \ac{CSS} und \ac{JS}-Dateien.
Zunächst wird sämtlicher Code Code der für die Darstellung des Templates innerhalb des $body$-Elementes definiert wurde entfernt. Lediglich die $script$-Tags bleiben innerhalb des $body$s bestehen, so dass sämtliche vordefinierte \ac{JS}-Dateien weiterhin geladen werden.
\begin{figure}[p]
\centering
\includegraphics[scale=0.32]{img/index-grundaufbau-1.png}
\includegraphics[scale=0.32]{img/index-grundaufbau-2.png}
\caption{Grundaufbau der $index.html$, um die Elm-Applikation zu injizieren}\label{fig:index-grundaufbau}
\end{figure}

\begin{figure}[p]
\centering
\includegraphics[scale=0.5]{img/elm-grundaufbau-start.png}
\caption{Grundgerüst der Elm-Applikation}\label{fig:elm-grundaufbau-start}
\end{figure}
Damit in das Grundgerüst die Elm-Applikation geladen werden kann, ist ein weiterer $script$-Tag notwendig, der die kompilierte Elm-Applikationsdatei $elm.js$ einbindet. Des Weiteren muss die Elm-Applikation explizit aufgerufen und gestartet werden. Die Abbildung~\ref{fig:index-grundaufbau} zeigt die $index.html$ mit den notwendigen Änderungen in Zeile $22$, in welcher die kompilierte Elm-Applikation eingebunden wird, sowie den Zeilen $43$ bis $46$, in denen die Applikation gestartet wird und ein Ziel-\ac{HTML}-Element übergeben bekommt. Damit die Datei $elm.js$ eingebunden werden kann, muss sie zunächst erstellt werden. Dafür wird der Elm-Compiler $elm-make$ mit dem Zusatz $Main.elm\,--output\,elm.js$ genutzt. Hierbei wird die Datei $Main.elm$ kompiliert und das Ergebnis in die Datei $elm.js$ gespeichert.
Es bestehen insgesamt drei Möglichkeiten die Elm-Applikation innerhalb der $index.html$ aufzurufen:
\begin{enumerate}
\item{$fullscreen$}: Der erzeugte Code der Applikation wird in den $body$-Tag einer \ac{HTML}-Datei geladen und überschreibt den sonstigen \ac{HTML}-Code

\item{$embed$}: Der erzeugte Code der Applikation wird in den übergebenen DOM-Knoten geladen

\item{$worker$}: Initialisiert die Applikation ohne grafische Benutzeroberfläche
\end{enumerate}
Die Abbildung~\ref{fig:index-grundaufbau} zeigt, dass für dieses praktische Beispiel Version $2$ genutzt wird. Diese Version bietet den Vorteil, dass eine Elm-Applikation gezielt in einen vordefinierten Bereich einer gesamten Webseite platziert werden kann. Auf diese Weise kann eine Elm-Applikation problemlos in eine bestehende Webseite eingebaut und erweitert werden. Ferner erlaubt diese Art der Injizierung zusätzliche externe \ac{JS}- und \ac{CSS}-Dateien hinzuzufügen.
Nutzt man beispielsweise Version $1$, so ist die Elm-Applikation im absoluten Vordergrund und lässt die Interaktion mit anderen Elementen die zuvor auf der Webseite definiert wurden nicht mehr zu. Version $3$ wiederum erzeugt keine grafische Benutzeroberfläche, wodurch ebenso keine Interaktion möglich ist.
Wahlweise besteht die Möglichkeit über den mitgelieferten Elm-Webserver $elm-reactor$ die Applikation manuell zu starten. Jedoch wird die Elm-Applikation dabei als $fullscreen$-Applikation gestartet und ist für die Zwecke dieser Arbeit nicht geeignet, weswegen auf die manuell Kompilierung zurückgegriffen werden muss.

Zusätzlich zum Grundgerüst der $index.html$ muss nun noch das Grundgerüst der eigentlichen Elm-Applikation erstellt werden. Wie im Kapitel \nameref{sec:Konzept} beschrieben, ist Elm nach einem Model-View-Update-Konzept aufgebaut. Entsprechend sind das die drei notwendigen Funktionen, die es zu realisieren gilt, damit die Elm-Applikation lauffähig ist. Um \ac{HTML}-Code zu erzeugen gibt es das Elm-Paket $elm-lang/html$. Es liefert einerseits die Funktionen um \ac{HTML}-Elemente zu erzeugen, andererseits drei Funktionen $beginnerProgram$, $program$ und $programWithFlags$. Diese kümmern sich um die Bereitstellung und Auslieferung der Applikation, so dass sich Entwickler ganz auf die eigentliche Programmierung konzentrieren können. Dabei variieren stets die Übergabeparameter, wodurch die Applikation leicht erweitert und komplexer werden kann. So verlangt die Funktion $beginnerProgram$ nur die bekannten Funktionen $model$, $view$ und $update$ als Übergabeparameter. Hierbei können jedoch keine asynchronen Funktionen wie \ac{HTTP}-Requests genutzt werden.
Dafür gibt es wiederum die erweiterte Funktion $program$, die als vierten Übergabeparameter sogenannte $subscriptions$ erwartet. Sie werden für die Kommunikation zwischen Elm und \ac{JS}, sowie Verbindungen zu Websockets genutzt.
Die dritte und letzte Möglichkeit der Initialisierung ist die Funktion $programWithFlags$. Hierbei wird die Übergabe eines initialen $Model$s an die Elm-Applikation ermöglicht, um den Zustand der Applikation dynamisch setzen zu können.
\begin{figure}[ht]
\centering\includegraphics[scale=0.6]{img/programWithFlags_pass_data.png}
\caption{Eine beispielhafte Initialisierung der Elm-Applikation mit initialen, dynamischen Werten}\label{fig:programWithFlags}
\end{figure}
Die Abbildung~\ref{fig:programWithFlags} zeigt beispielhaft die Implementierung der $Html.App.programWithFlags$-Funktion. Der zweite Übergabeparameter an die $Elm.Main.embed$-Funktion ist dabei ein $Record$ mit den gewünschten initialen Werten.
Für die Umsetzung dieses praktischen Beispiels ist der Grundaufbau wie in Abbildung~\ref{fig:elm-grundaufbau-start} notwendig. Dies sind die minimal notwendigen Funktionen, um in den weiteren Schritten die einzelnen Sektionen des Views von \ac{HTML} nach Elm zu portieren, in einzelne Funktionen auszulagern und letzten Endes der Versuch, die gesamte Applikation zu modularisieren.
Aus der Abbildung~\ref{fig:elm-grundaufbau-start} wird ersichtlich, dass zunächst das gesamte Paket $elm-lang/html$ importiert wird. Die vorherige Installation des Pakets ist analog zu der in der Abbildung~\ref{fig:elm-install-package} beschriebene Vorgehensweise. Des Weiteren wird der Grundaufbau der Elm-Applikation ersichtlich.
Innerhalb der $main$-Funktion werden dabei alle notwendigen Funktionen an die $HTML$-Bibliothek weitergereicht. Dazu gehören die $init$, $view$, $update$ und $subscriptions$-Funktionen. Wie der Name bereits suggeriert initiiert die $init$-Funktion ein $Model$. Dabei wird dieses entweder anhand der dynamisch übergebenen Werte, oder durch ein Ausweich-Model ($initialModel$) erstellt, falls keine Daten übergeben wurden.
Die Funktion $view$ erzeugt in der Abbildung~\ref{fig:elm-grundaufbau-start} zunächst ein $div$-Element, während die $update$-Funktion auf jede eingehende Interaktion ein Tupel mit einem unveränderten $model$, sowie keinem $Effekt$ zurück gibt. Um die Applikation aufzurufen muss die Datei $index.html$ im Browser geöffnet werden.


\subsection{Überführung des Views}
\label{sec:ueberfuehrung-view}
Nachdem der Grundaufbau der Elm-Applikation beidseitig ausgeführt wurde, können nun die einzelnen Sektionen der originalen $index.html$ aus dem $Agency$-Template nativ in Elm überführt werden. Dafür wird das Online-Tool $html-to-elm$ auf der Webseite \url{http://mbylstra.github.io/html-to-elm/} zur Hilfe genommen. Der gesamte zuvor im $body$-Tag befindliche \ac{HTML}-Code wird kopiert und zur Konvertierung in das entsprechende Textfeld des Tools eingefügt. Der konvertierte Elm-Code sollte daraufhin auf der gegenüberliegenden Seite zu sehen sein. Durch einen Klick auf $copy$ wird der erstellte Elm-Code in die Zwischenablage kopiert, so dass der Inhalt in die $view$-Funktion der $Main.elm$-Datei eingefügt werden kann. Der kopierte Text fungiert dabei als kompletter Rückgabewert der $view$-Funktion.
\begin{figure}[h]
\centering
\includegraphics[scale=0.5]{img/elm-naming-error.png}
\caption{Elm-Compilerfehler bei fehlenden Funktionen im Namensraum}\label{fig:elm-naming-error}
\end{figure}
Das Tool $html-to-elm$ sollte den Code bereits nach den Regeln des Style-Guides formatiert haben, so dass hiermit der erste Schritt der Portierung abgeschlossen ist. Um Namenskonflikte im aktuellen Namensraum zwischen Funktionen zu verhindern, werden innerhalb der importierten $HTML$-Bibliothek die genutzten Funktionen zur Erstellung von \ac{HTML}-Code explizit genannt. Der Compiler sollte nach dem Entfernen des Kommandos $exposing (..)$ eine Warnung ausgeben und auf die fehlenden Funktionen hinweisen, wie es in Abbildung~\ref{fig:elm-naming-error} gezeigt wird. Schlägt der Compiler dabei vor, die Funktion aus dem Namensraum $Html.Attributes$ aufzurufen, muss sie entsprechend in die Liste der zu ladenden Funktionen der $Attribute$ hinzugefügt werden, wie in Abbildung~\ref{fig:elm-grundaufbau-start} in Zeile 3. Die Funktionen zur Generierung von \ac{HTML}-Tags hingegen sollten in die Liste der $import\,Html$s stehen. Nachdem alle Funktionen korrekt in den aktuellen Namensraum aufgenommen wurden, sollten keinerlei Compilerfehler auftreten.


\subsection{Beheben von JavaScript-Fehlern}
\label{sec:javascript-errors}
Das Öffnen der Seite mitsamt der Developer-Tools von Google-Chrome verrät, dass die bisher eingebundenen \ac{JS}-Dateien nicht fehlerfrei funktionieren. Nutzt man das Template wie vorgesehen, ist sämtliche Funktionalität vorhanden. Bei der Überführung des Templates in Elm gilt es allerdings zu beachten, dass die Elm-Applikation den $view$ ausliefert und diesen erst verzögert darstellt. Dem menschlichen Betrachter fällt dies zwar nicht auf, jedoch greifen die \ac{JS}-Skripte auf ein noch nicht geladenes Element zu. Es ist dementsprechend notwendig, die einzelnen Skripte anzupassen und mit Hilfe der $\$(document).ready()$-Funktion des Frameworks $jQuery$ (\url{https://learn.jquery.com/using-jquery-core/document-ready/}) erst nach dem kompletten Seitenaufbau zu laden.
Innerhalb der Datei $agency.js$ muss nicht nur die Funktion $\$('a.page-scroll').bind(...)$, sondern auch die nachfolgenden Funktionen innerhalb des $\$(function(){..});$-Blocks liegen. Die \ac{JS}-Datei $cbpAnimatedHeader.js$ hingegen erzeugt einen Fehler, wie in Abbildung~\ref{fig:js-classie-error} zu sehen ist.
\begin{figure}[h]
\centering
\includegraphics[scale=0.4]{img/error-javascript-classie.png}
\caption{JavaScript-Fehler innerhalb der Developer-Tools}\label{fig:js-classie-error}
\end{figure}
Um diesen zu beheben, muss der gesamte Dateiinhalt in eine $\$(document).ready()$-Funktion geschachtelt werden. Zusätzlich dazu ist es notwendig, dass die Codezeile wie sie in Abbildung~\ref{fig:code-to-add} zu sehen ist, als erstes in der Funktion $scrollPage()$ hinzugefügt wird.
\begin{figure}[h]
\centering
header = document.querySelector( '.navbar-fixed-top' );
\caption{Notwendiger Code, um JavaScript-Fehler zu beheben.}\label{fig:code-to-add}
\end{figure}
Der Code hat zur Folge, dass das Navigations-Element ausgewählt und an die $classie$-Funktion weitergegeben wird. Fehlt diese Zeile, erfolgt ein Aufruf der $classie$-Funktion mit einem undefinierten Wert, wodurch der Fehler in Abbildung~\ref{fig:js-classie-error} zustande kommt. Nachdem diese beiden Fehler behoben wurden, kann nun die $Scroll-Spy$-Funktionalität auf der Webseite betrachtet werden. Scrollt der Nutzer nun auf oder ab und befindet sich dabei in einer innerhalb der Navigation definierten Sektion, so wird das entsprechende Gegenstück in der Navigationsleiste farblich hinterlegt. Des Weiteren verkleinert sich die Navigationsleiste, wenn der Nutzer eine bestimmte Entfernung nach unten gescrollt hat.


\subsection{Auslagern des Views}
\label{sec:auslagern-des-views}
Um die einzelnen Sektionen der Webseite im Quellcode klar voneinander zu trennen, werden die einzelnen Sektionen in eine jeweils eigene Funktion ausgelagert.
Für jede Sektion wird hierfür eine gleichnamige Funktion angelegt, die dem Muster in Abbildung~\ref{fig:elm-view-section} folgt. Die $id$ eines \ac{HTML}-Elementes könnte entsprechend für den Funktionsnamen $nameDerSektion$ genutzt werden, da dieses Attribut ohnehin einzigartig in einem gesamten \ac{HTML}-Dokument vorkommt und eine klare Namensgebung liefert.
\begin{figure}[htb]
\centering
\includegraphics[scale=0.53]{img/elm-html-sections.png}
\caption{Deklaration einer Sektion des Views in Elm}\label{fig:elm-view-section}
\end{figure}
Am Beispiel der $portfolio$-Sektion des zu überführenden Templates sehen die daraus resultierende Funktion teilweise aus wie in Abbildung~\ref{fig:elm-view-section-function}. Die einzelnen Klassen und ID`s wurden beibehalten und übernommen. Nachdem die einzelnen Sektionen in Funktionen ausgelagert wurden fällt auf, dass die Sektionen nicht mehr angezeigt werden. Dafür ist es notwendig die ursprüngliche $view$-Funktion mit den Funktionsaufrufen der Sektionen zu versehen. Dieses Vorgehen kann in der Abbildung ~\ref{fig:elm-view-sections-call} betrachtet werden. Das Resultat der $view$-Funktion ist unverändert, ist jedoch nun im Code übersichtlicher.
\begin{figure}[h]
\centering
\includegraphics[scale=0.41]{img/elm-view-portfolio-section-function.png}
\caption{Ausgelagerter View in eine eigene Funktion}\label{fig:elm-view-section-function}
\end{figure}

\begin{figure}[h]
\centering
\includegraphics[scale=0.4]{img/elm-view-section-functions-calling.png}
\caption{Aufruf der einzelnen Sektionen im View}\label{fig:elm-view-sections-call}
\end{figure}


\subsection{Hinzufügen von Klick-Events}
\label{sec:erweiterung-clicks}
Der derzeitige Stand der überführten Applikation stellt bereits sämtliche Sektionen auf der Webseite dar und erlaubt das Öffnen eines Modals durch einen Klick auf eines der Portfolio-Elemente. Bei einem Klick auf ein Element in der Navigationsleiste springt der Viewport des Browsers jedoch ohne jegliche Verzögerung direkt auf die ausgewählte Sektion. Das ursprüngliche Template hingegen nutzt die Funktionalität $smooth-page-scrolling$, wodurch der Nutzer einen sichtbaren Scroll-Effekt erhält, während automatisch zur ausgewählten Sektion gescrollt wird. Innerhalb der Datei $agency.js$, in der vorab bereits die Funktion des $Scroll-Spy$s wiederhergestellt wurde, wird in den Zeilen $9$ bis $15$ die Funktionalität des automatischen scrollens bei einem Klick umgesetzt, wie in Abbildung~\ref{fig:non-functional-scrolling} zu sehen ist.
\begin{figure}[h]
\centering
\includegraphics[scale=0.37]{img/non-functional-scrolling}
\caption{Funktion zum scrollen bei einem Klick}\label{fig:non-functional-scrolling}
\end{figure}
Dabei wird zunächst ein $Ereignisbehandler$ an die Link-Elemente mit der Klasse $page-scroll$ gebunden. Klickt der Nutzer eines dieser Elemente, so wird die darauf folgende Funktion ausgeführt. Diese Funktion wiederum ließt Informationen des angeklickten Elementes aus, insbesondere das $href$-Attribut, welches die Id zur angeklickten Sektion enthält. Mittels der $animate$-Funktion wird letztlich zur gewünschten Sektion gescrollt.
Diese Funktionalität ist nicht verfügbar, da das externe \ac{JS} versucht auf \ac{HTML}-Elemente innerhalb der Elm-Applikation zuzugreifen. Um dieses Problem zu lösen, kann einerseits die Funktion innerhalb der Datei $agency.js$ umgeschrieben werden, so dass sämtliche Klicks innerhalb des $body$-Elementes abgefangen werden, da sich dieser nicht Teil der Elm-Applikation ist. Andererseits kann der Klick innerhalb der Elm-Applikation registiert und über sogenannte Ports nach außen getragen werden. In diesem Beispiel ist das die bevorzugte Methode, da so eine klare Trennung zwischen der Elm-Applikation und den externen \ac{JS}-Skripten herrscht.
Die Elm-Bibliothek $Html.Events$ enthält die Methode $onWithOptions$, mit Hilfe derer ein Klick auf ein Element des $views$ in Elm eine $Msg$ erzeugt. Diese $Msg$ wird an die $update$-Funktion weitergeleitet und kann dort entsprechend ausgewertet werden.
\begin{figure}[htb]
\centering
\includegraphics[scale=0.3]{img/elm-port.png}
\caption{Definition eines neuen Union Types, sowie eines Ports für die Kommunikation zwischen Elm und \ac{JS}}\label{fig:create-port}
\end{figure}
Bislang gibt es in der Elm-Applikation noch keine $Msg$-Typen, welche die $update$-Funktion verarbeiten könnte. Dementsprechend wird zunächst ein neuer Union-Type $Msg$ mit dem Eintrag $ClickedElem\,String$ angelegt. Der Union-Type muss nun noch in der $update$-Funktion behandelt werden. Hierbei wird ein neuer $case$ angelegt, mit dem zuvor erstellten Eintrag $ClickedElem$ und dem Übergabeparameter $elem$ vom Typ $String$. Der vorherige Eintrag "$\_ ->$", welcher für alle Fälle gilt, kann entfernt werden. Der Rückgabewert des $ClickedElem$-Falles ist dabei ein Tupel, bestehend aus dem unveränderten $model$, sowie einem Effekt. In diesem Fall ist der Effekt das Senden der ausgewählten Elemente nach außen zu einem \ac{JS}-Skript, das die Daten verarbeitet. Der Effekt wird dabei über den Port $sendClickedElemToJs$ nach außen geschickt. Der Port muss definiert werden mit dem Kommando in Zeile $21$ aus der Abbildung~\ref{fig:create-port}. Das Kommando ist lediglich die Signatur der gewünschten Kommunikationsweise und wird durch den Elm-Compiler während des Kompiliervorgangs entsprechend verarbeitet.
Beim Versuch den Code wie er in Abbildung~\ref{fig:create-port} zu sehen ist zu kompilieren, wird der Compiler mit der Fehlermeldung "You are declaring port $sendClickedElemToJs$ in a normal module" antworten. Die Fehlermeldung bedeutet, dass das Modul in dem ein Port definiert wird explizit als ein Port-Modul deklariert werden muss. Die erste Zeile des Hauptmoduls muss folglich noch mit dem Schlüsselwort $port$ versehen werden. Damit die Elm-Applikation die Funktion $ClickElem$ aufruft, bedarf es noch des Aufrufes der zuvor erwähnten $onWithOptions$-Methode. Diese wird in den $view$ an den entsprechenden Stellen eingesetzt. Die Links der Navigationsleiste befinden sich in der Funktion $navigation$ und müssen um den Zusatz der $onWithOptions$-Funktion, sowie den Übergabeparametern $ClickedElem$ und dem $href$-Wert erweitert werden. In Abbildung~\ref{fig:elm-view-onclick} kann beispielhaft die Erweiterung an einigen Stellen begutachtet werden. Die Funktion $onWithOptions$ bekommt dabei zunächst einmal das $Event$, auf das sie einen Ereignishandler setzen soll. In diesem Fall ist dies ein $click$. Darauf folgt ein optionaler $Record$ von Typ $Options$. Hier müssen die Werte $stopPropagation$ und $preventDefault$ auf $True$ gesetzt werden. Fehlt diese Einstellung, so wird das scrollen sofort ausgeführt und kann möglicherweise andere Ergebnisbehandler in denen der Link geschachtelt ist ausführen. Dies würde zu einem undefinierten Verhalten führen. Der letzte Parameter ist die die Decoder-Funktion $Json.succeed$. Sie führt dazu, dass die übergebene Funktion $ClickedElem$ direkt und ohne weitere Überprüfungen ausgeführt wird.
\begin{figure}[hbt]
\centering
\includegraphics[scale=0.4]{img/elm-view-onclick.png}
\caption{Erweiterung des Views um einen $onWithOptions$ Ereignisbehandler in Elm}\label{fig:elm-view-onclick}
\end{figure}
Seitens der Elm-Applikation ist die notwendige Erweiterung nun abgeschlossen. Jedoch muss noch die $index.html$ um eine sogenannte $subscription$ erweitert werden. Das bedeutet, dass innerhalb der $index$-Datei die Nachricht der Elm-Applikation des angeklickten Elements entgegengenommen und verarbeitet werden muss. Diesbezüglich wird in der $index.html$ nach der Codestelle die für die Initialisierung der Elm-Applikation zuständig ist die Funktion aus Abbildung~\ref{fig:scrollTop} hinzugefügt.
\begin{figure}[hbt]
\centering
\includegraphics[scale=0.4]{img/scrollTop.png}
\caption{Entgegennahme der Daten aus der Elm-Applikation und anschließendes scrollen}\label{fig:scrollTop}
\end{figure}
Durch das Kommando $app.ports.sendClickedElemToJs.subscribe$ werden sämtliche Daten die Elm über den Port $sendClickedElemToJs$ sendet entgegengenommen und die nachfolgende Funktion ausgeführt. Diese Funktion wurde in diesem Beispiel lediglich aus der Datei $agency.js$ kopiert. Sie ermittelt das \ac{HTML}-Element im Baum anhand der übergebenen Id und scrollt innerhalb von zwei Sekunden zur obersten Stelle des übergebenen Elements.

\subsection{Modularisierung der Applikation}
\label{sec:modularisierung-der-applikation}
Durch das Auslagern aller einzelnen Sektionen der darzustellenden \ac{HTML}-Elemente ist die daraus resultierende $Main.elm$-Datei recht unübersichtlich. Abhängigkeiten zwischen einzelnen Funktionen sind kaum mehr erkennbar und die Programmlogik ist nicht unterscheidbar von Funktionen, die für die Darstellung des Views zuständlich sind. Folglich ist es sinnvoll einzelne Teile der Applikation in mehrere Dateien und Ordner zu verschieben. Eine solche Strukturierung hilft dabei, die womöglich fehlerbehafteten Teile der Applikation schneller zu finden und die einzelnen Bereiche der Applikation deutlicher voneinander zu trennen. Dabei werden die einzelnen Programmteile des Views ausgelagert in eigene Module. Dasselbe Prinzip wird für das Model und die Update-Funktion angewandt. Im Gegenzug sollen die ausgelagerten Programmteile global im Hauptmodul importiert und an den entsprechenden Stellen aufgerufen werden.

\subsubsection{View}
\label{sec:auslagern-view}
Jede Funktion die bisher eine Sektion des Views erzeugt hat, wird in eine neu angelegte $.elm$-Datei verschoben und die Funktion umbenannt zu $view$. Die Datei bekommt dabei den Namen, den die Funktion vorher trug. Ferner wird ein Ordner $View$ erzeugt, in den all diese Dateien hin verschoben werden. Damit die neuen Dateien als Module eingebunden werden können, müssen sie als solches definiert werden. Das bedeutet, dass jede Datei mit dem Kommando $module\,View.NameDerDatei\,exposing\,(view)$ initialisiert werden muss. Der Zusatz $view$ nach dem Kommando $exposing$ gibt an, dass die Funktion $view$ nach außen hin sichtbar ist und vom importierenden Modul genutzt werden kann.
\begin{figure}[hbt]
\centering
\includegraphics[scale=0.5]{img/imports-main.png}
\caption{Einbindung und Aufruf der ausgelagerten View-Funktionen}\label{fig:elm-main-view-imports}
\end{figure}
Nachdem alle Funktionen in ein eigenes Modul ausgelagert wurden, müssen alle Module im Hauptmodul $Main.elm$ importiert werden. Des Weiteren müssen die $view$-Funktionen der einzelnen Module im Hauptmodul aufgerufen werden. Die Abbildungen~\ref{fig:elm-main-view} und \ref{fig:elm-main-view-imports} zeigen die $view$-Funktion des Hauptmoduls, sowie den Aufruf um die angelegten Module zu importieren.
\begin{figure}[hbt]
\centering
\includegraphics[scale=0.6]{img/view-main.png}
\caption{Einbindung und Aufruf der ausgelagerten View-Funktionen}\label{fig:elm-main-view}
\end{figure}
Letzten Endes müssen die einzelnen Module noch mit den notwendigen $import$s für die verwendeten Funktionen erweitert werden. Das Hauptmodul wiederum kann von den nicht benutzten $Html$- beziehungsweise $Html.Attributes$-Funktionen gesäubert werden.

\subsubsection{Update}
\label{sec:auslagern-update}
Nicht nur die für die Darstellung verantwortlichen Funktionen sollten modularisiert werden, sondern auch die Programmlogik. Hierfür wird ein weiterer Ordner $Update$ erzeugt. Die bisherige $update$-Funktion wird demzufolge analog zu den $View$-Funktionen ausgelagert in ein eigenes Modul im neu erzeugten Ordner $Update$. Das $Update$-Modul folgt derselben Namensgebung wie ein einzelner $View$ und wird definiert als $Update.Update$. Zusätzlich müssen auch an dieser Stelle die $import$s der benutzten Pakete übernommen und definiert werden, wobei das Hauptmodul diese entfernen kann.

\subsubsection{Model}
\label{sec:auslagern-Model}
Letztlich wird noch das $model$, das sämtliche Daten die den Status der Applikation beschreiben enthält, in ein eigenes Modul im Unterordner $Model$ überführt. Die Einbindung dieses Moduls funktioniert analog zur Modularisierung von $Update$ und $View$.
Mit Hilfe dieser Modularisierung wird das angestrebte \ac{MVU}-Konzept von Elm besonders deutlich.


\subsection{Asynchrones Laden von Daten}
\label{sec:async-laden}
Klickt man innerhalb der Portfolio-Sektion der Webseite auf ein Element, öffnet sich ein Modal in dem weitere Informationen dargestellt werden. Die bestehende Elm-Applikation wird nun um das Feature des asynchronen Ladens von Informationen erweitert, die anschließend in dem geöffneten Modal angezeigt werden. Beispielhaft wird hier die \ac{API} von \ac{ICNDb} (\url{https://api.icndb.com}) genutzt. Sie liefert bei jeder Anfrage einen zufälligen Witz zurück.
Für dieses Funktionalität muss zunächst das $model$ erweitert und angepasst werden, da dies die einzige Möglichkeit in einer Elm-Applikation ist, Daten beziehungsweise den Status der Applikation zu speichern. Das $model$ bekommt entsprechend ein weiteres Feld $async\_content$ vom Typ $String$.
Bei einem Klick auf einen der Portfoliobeiträge soll ein Modal geöffnet und ein zufälliger Witz präsentiert werden. Dafür wird über die externe \ac{API} ein zufälliger String angefordert, vom Server generiert und letztlich an die Elm-Applikation zurückgegeben. Ebenso wäre es möglich einen Server für das Backend zu erstellen, auf dem eine Datenbank läuft, so dass Daten asynchron angefordert werden können. Das asynchrone Anfordern von Daten basiert sowohl bei einem lokalen, wie auch externen Server auf dem gleichen Konzept, weswegen dieser Schritt entfällt und der vorhandene externe Service genutzt wird.
Ein solcher asynchroner Request stellt im Grunde eine Verletzung des Konzeptes von Elm dar, dass es keinerlei Seiteneffekte gibt. Da nicht bekannt ist, wann der Request endet und welchen Status die Antwort besitzt, ist zunächst nicht vorhersehbar, wie der Status der Applikation nach dem Request aussehen wird. Um dieses Problem zu vermeiden, ist es notwendig alle möglichen Fälle, sprich den Fall einer erfolgreichen, sowie fehlerhaften Übertragung, zu behandeln. Auf diese Weise ist gewährleistet, dass die Applikation sich nicht plötzlich in einem undefinierten Zustand befindet. Es bedarf zusätzlich der Bibliotheken $Http$, $Json.Decode$ und $Task$, damit ein asynchroner Request ausgeführt werden kann. Die Bibliotheken müssen installiert und in die jeweiligen Module importiert werden. Des Weiteren gilt es den Klick auf den Portfoliobeitrag abzufangen, um den Request zu starten und das Ergebnis in den View einzuarbeiten. Dafür gibt es die $onClick$ Funktion aus der $Html.Events$-Bibliothek. Sie bekommt eine auszuführende Funktion aus dem $Update$-Modul als Parameter, in diesem Fall $GetRandomString$, wie es in Abbildung~\ref{fig:portfolio-async} zu sehen ist. Bei einem Klick auf das Element wird eine $Msg$ von diesem Typ erstellt und muss in der $update$-Funktion behandelt werden.
\begin{figure}[h]
\centering
\includegraphics[scale=0.4]{img/portfolio-async.png}
\caption{OnClick-Ereignisbehandler in Elm}\label{fig:portfolio-async}
\end{figure}
Wie zuvor bei der Definition einer $Msg$ um Daten durch Ports nach außen zu kommunizieren, muss auch die Nachricht $GetRandomString$ als $Msg$-Typ deklariert werden, um so in der $update$-Funktion behandelt werden zu können. Außerdem müssen die möglichen Statusfälle des Requests behandelt und als $Msg$ hinzugefügt werden. Abbildung~\ref{fig:msg-async} zeigt die hinzugefügten $Msg$-Typen.
\begin{figure}[h]
\centering
\includegraphics[scale=0.5]{img/msg-async.png}
\caption{$Msg$-Deklaration mit asynchronem Request}\label{fig:msg-async}
\end{figure}
Nachdem die neuen Interaktionsmöglichkeiten in der Elm-Applikation implementiert wurden, muss nun der eigentliche Request ausführbar gemacht werden. Die Funktion $GetRandomString$ soll dabei ein unverändertes Model, sowie eine asynchron auszuführende Funktion in Form eines Effektes als Tupel zurückliefern. Dabei ist die asynchrone Funktion der eigentliche \ac{HTTP}-Request, wie er in Abbildung~\ref{fig:fetchasync-async} erkennbar ist.
\begin{figure}[h]
\centering
\includegraphics[scale=0.3]{img/fetchasync-async.png}
\caption{Asynchroner \ac{HTTP}-Request in Elm}\label{fig:fetchasync-async}
\end{figure}
Dabei wird die Funktion $Http.get$ genutzt, um einen Request an die $url$ auszuführen. Die Antwort des Requests wird an die Funktion $decodeAnswer$ gereicht, welche das eingehende \ac{JSON}-Objekt dekodiert und die gewünschten Informationen herausfiltert. Die Dekodierfunktion kann in Abbildung~\ref{fig:decodeanswer-async} betrachtet werden. Sie filtert das geschachtelte \ac{JSON}-Objekt, so dass nur der gewünschte String übrig bleibt.
\begin{figure}[h]
\centering
\includegraphics[scale=0.4]{img/decodeanswer-async.png}
\caption{Dekodierung eines Json-Objektes in Elm}\label{fig:decodeanswer-async}
\end{figure}
Abhängig vom Status des Requests, wird der jeweilige Fall in der $update$-Funktion aufgerufen. Bei einer erfolgreichen Übertragung wird die Funktion $FetchSucceed$ ein neues Model mit dem asynchron abgerufenen String zurückliefern. Der String wird dabei in das Feld $async\_content$ des $model$ geschrieben. Bei einer fehlgeschlagenen Übertragung wird die Funktion $FetchFail$ das unveränderte $model$ zurückgeben. Alternativ kann auch eine mögliche Fehlermeldung in das $model$ eingearbeitet und im $view$ dargestellt werden.
\begin{figure}[h]
\centering
\includegraphics[scale=0.4]{img/update-async.png}
\caption{$Update$-Fälle für den asynchronen Request}\label{fig:update-async}
\end{figure}
Die fertige $update$-Funktion mit allen abgehandelten Fällen ist in Abbildung~\ref{fig:update-async} dargestellt. Der asynchrone Request wird dabei durch $GetRandomString$ initiiert. Die Manipulation des $model$ findet lediglich in der $FetchSucceed$-Funktion statt.
Nachdem die Elm-Applikation an dieser Stelle fehlerfrei kompiliert wurde, kann der asynchrone Request auf der Webseite getestet werden. Mit Hilfe der Developer-Tools kann die ausgehende Anfrage an den \ac{ICNDb}-Server beobachtet und ausgewertet werden. Auch die Antwort des Servers kann unbehandelt in Abbildung~\ref{fig:request-async} begutachtet werden.
\begin{figure}[htb]
\centering
\includegraphics[scale=0.4]{img/request-async.png}
\caption{Ausgehender HTTP-Request und eingehende Antwort}\label{fig:request-async}
\end{figure}
Die Abbildung zeigt ferner wie der zurückgelieferte Text des Requests fehlerfrei dekodiert und in das $model$ übertragen, sowie letztlich in der $view$-Funktion des $Popup$-Moduls mittels des Kommandos $text\,model.async\_content$ angezeigt wird.



\subsection{Beobachtungen}
\label{sec:Beobachtungen}
In diesem Abschnitt sollen Auffälligkeiten, die während der Überführung des $agency$-Templates in eine native Elm-Applikation auftraten, aufgezeigt und erläutert werden. Es handelt sich hierbei nicht nur um Probleme, sondern auch um interessante Erkenntnisse im Hinblick auf die Unterschiede zwischen herkömmlicher Webentwicklung mit \ac{HTML}, \ac{CSS} und \ac{JS} gegenüber Elm, die es zu erwähnen gilt.

\subsubsection{Natives Einbinden von CSS in Elm nur beschränkt möglich}
Herkömmlicherweise können Elemente einer \ac{HTML}-Datei auf drei Arten gestylt werden.
\begin{enumerate}
\item{Inline \ac{CSS}:}
	\subitem Das Inline \ac{CSS} erlaubt Design-Anpassungen für genau ein bestimmtes Element. Der verfasste \ac{CSS}-Code kann nicht von anderen Elementen genutzt werden und wird über das \ac{HTML}-Attribut $style$ eingeführt.
\item{Internes \ac{CSS}:}
	\subitem \ac{HTML} enthält den $style$-Tag. In diesem kann natives \ac{CSS} verfasst, bestimmte Klassen oder Id's erstellt und sämtliche gängigen Design-Anpassungen durchgeführt werden.
\item{Externes \ac{CSS}:}
	\subitem Die Erweiterung des internen \ac{CSS} ist das externe \ac{CSS}, bei dem der gesamte \ac{CSS}-Code ausgelagert wird in eine eigene $.css$-Datei. Diese wird wiederum mit Hilfe des $link$-Tag der \ac{HTML}-Datei eingebunden.
\end{enumerate}
In Elm ist es nicht möglich nativen \ac{CSS}-Code zu verfassen. Jedoch kann mittels der Bibliothek $elm-lang/html$ inline \ac{CSS} erzeugt werden. Das Design wird dabei nativ in Elm über die $Html.style$-Funktion programmiert. Ferner ist das erstellen von $style$-Tags in Elm nicht möglich, wodurch auch die Möglichkeit für internes \ac{CSS} entfällt. Es ist zwar praktisch möglich externe \ac{CSS}-Dateien in die Elm-Applikation nativ in Elm einzubinden, jedoch stört das immens die Nutzbarkeit der Applikation. Die Elm-Applikation wird zunächst ohne das \ac{CSS} geladen, während die externe Datei nachgeladen wird. Das bedeutet für den Nutzer der Webseite, dass zunächst nur Texte und Bilder ohne Styling angezeigt werden. Nachdem die externe Datei vollständig übertragen wurde, wird das darin enthaltene Design auf die \ac{HTML}-Elemente angewandt. Der zeitliche Abstand zwischen initialem Laden und der Anwendung des Stylesheets hat ein sichtbares $flackern$ zur Folge, wodurch eine Nutzung in einem fertigen System entfällt. Dies ist der Grund, weswegen bereits bestehende \ac{CSS}-Dateien des Templates nicht nativ in Elm überführt, sondern über das \ac{HTML}-Grundgerüst eingebunden wurden. Zusätzlich zu diesem Umstand wird inline- und internes \ac{CSS} nicht vom Browser zwischengespeichert. Das hat eine längere Ladezeit zur Folge und sollte vermieden werden. Des Weiteren ist es nur bedingt möglich \ac{CSS}-Code in Elm an mehreren Stellen zu verwenden, wie es üblicherweise mit in \ac{CSS} definierten Klassen der Fall ist.


\subsubsection{HTML-Code ist in Elm wesentlich kürzer}
\ac{HTML}-Tags mit den gleichnamigen Funktionen nativ in Elm zu erstellen ist einerseits signifikant kürzer, andererseits viel übersichtlicher als das Verfassen von herkömmlichen \ac{HTML}-Code.
\begin{figure}[h]
\centering
HTML:
<div>
</div>\\
Elm:
div[$\,$][$\,$]
\caption{Ein $div$-Element in \ac{HTML} und Elm}\label{fig:div-element}
\end{figure}
Die Abbildung~\ref{fig:div-element} zeigt ein $div$-Element, wie es einerseits in klassischem \ac{HTML}, andererseits in Elm realisiert wird. In Elm handelt es sich dabei um einen Funktionsaufruf mit zwei Parametern, in diesem Fall zwei leeren Listen. Ein \ac{HTML}-Element hat dabei eine Anzahl von $5+2*n_0$ Zeichen, wobei $n_0$ gleich der Zeichenanzahl des Tag-Namens und $5$ die Zeichen für das öffnende, sowie schließende Tag sind. Am Beispiel des $div$-Elementes wäre $n_0:=3$, wodurch die Implementierung in \ac{HTML} $5+2*3$, sprich $11$ Zeichen benötigt.
In Elm hingegen benötigt die Erstellung eines \ac{HTML}-Elementes $4+n_1$ Zeichen, wobei $n_1$ gleich der Zeichenanzahl des Tags, sprich hier der Funktion $div$, und $4$ die Zeichenanzahl für die Parameterliste ist. Am selben Beispiel wären das $4+3$, sprich $7$ Zeichen. Die Konstanten $5$ (\ac{HTML}) und $4$ (Elm) können vernachlässigt werden, wodurch sich die dynamischen Werte $2*n_0$ und $n_1$ gegenüberstehen. Da alle \ac{HTML}-Tags in Elm mit einer gleichnamigen Funktion erstellt werden, kann $n_0$ gleich $n_1$ gestellt werden. Daraus folgt, dass Elm im Vergleich zu \ac{HTML} 50\% weniger Zeichen für die Erstellung eines \ac{HTML}-Elementes benötigt.
Die Differenz der Zeichen lässt sich auf das schließen der Tags in \ac{HTML} zurückführen. Nativer Elm-Code arbeitet mit Einrückungen, wohingegen klassischer \ac{HTML}-Code über die öffnenden und schließenden Tags geschachtelt wird. Dadurch entsteht bei \ac{HTML}-Code der Vorteil, dass ein gesamtes Dokument theoretisch in nur einer Zeile ohne Absätze definiert werden kann, während in Elm die Einrückungen und die damit verbundenen Zeilenumbrüche notwendig sind. Daraus folgt, dass Elm-Code zwangsweise übersichtlicher bleibt, da die Einrückungen nicht umgangen werden können und dem Entwickler eine visuelle Rückmeldung liefern. Der Elm-Compiler wird den Elm-Code nicht in \ac{JS}-Code umwandeln, sollten die Abstände inkorrekt sein. \ac{HTML}-Code bietet die Möglichkeit gleichermaßen übersichtlich zu programmieren, jedoch gibt es hier keinen Compiler oder sonstige Warnhinweise für den Fall, dass geschriebener Code unübersichtlich wird.
Durch die Einrückung in das damit verbundene Fehlen von schließenden Tags kam es zu weniger Flüchtigkeitsfehlern in Form von falschen Verschachtelungen oder dem simplen Fehlen der schließenden Tags bei einer tiefen oder über mehrere Absätze stattfindende Schachtelung.

\subsubsection{Elm hat eine stark wachsende Open-Source Gemeinschaft}
Trotzdem Elm erst im März 2012 entwickelt wurde, gibt es bereits eine Vielzahl an Gemeinschaften, um über die Sprache zu diskutieren, Hilfe zu erbitten oder gemeinsam an Projekten zu arbeiten. Neben einem eigenen Slack-Chatroom mit 2.701 angemeldeten Nutzern \cite[vgl.]{slack-user}, gibt es noch die offizielle Google-Gruppe. Offiziell gibt es derzeit $205$ Elm-Bibiliotheken \cite[vgl.]{elm-package}, von denen jede einzelne ein Open-Source-Projekt darstellt und für alle Entwickler freigegeben ist. Auf der Webseite \url{https://github.com/} gibt es zur Zeit 2.845 öffentliche Projekte \cite[vgl.]{elm-repositories} die komplett oder zumindest teilweise in Elm realisiert wurden. Ferner wurden 6.365 Probleme in diesen Projekten gemeldet, von denen 4.834 bereits gelöst und 1.531 noch geöffnet sind.  Alles in Allem wird deutlich, dass Elm eine wachsende Anhängerschaft besitzt und die Probleme der Webentwicklung auf eine andere Weise angeht. Wann immer Fragen auftraten konnten die Nutzer in einer der genannten Medien meist in wenigen Minuten aushelfen. Des Weiteren ist Elm anwendbar auf allen Betriebssystemen, die den \ac{NPM} unterstützen. Zusätzlich gibt es die öffentlichen Tools $elm-repl$, $elm-make$ und $elm-package$, mit denen eine Elm-Applikation umgesetzt werden kann. Darüber hinaus gibt es Quellen wie \url{http://guide.elm-lang.org/}, \url{http://elm-lang.org/get-started} und \url{http://www.elm-tutorial.org/en}, die mit Hilfe von Schritt-für-Schritt Anleitungen und praktischen Beispielen an die Programmiersprache heranführen.


\subsection{Auswertung}
\label{sec:Auswertung}
Nachdem die \ac{SPA} vollständig überführt und einige Beobachtungen erläutert wurden, steht die Auswertung der einzelnen Bewertungskriterien aus. Dieser Abschnitt erläutert dabei in welchem Maße die Kriterien erfüllt wurden und inwiefern die Erkenntnisse verallgemeinerbar sind und auf andere Webapplikationen zutreffen.
\begin{figure}[htb]
\centering
\begin{tabular}{ | p{8cm} | c | }
	\hline
	\textbf{Kriterium} & \textbf{Erfüllt}\\
	\hline
	\ref{sec:muster_wartbarkeit_und_lesbarkeit} \nameref{sec:muster_wartbarkeit_und_lesbarkeit} & \checkmark\\
	\hline
	\ref{sec:muster_zuverlaessigkeit} \nameref{sec:muster_zuverlaessigkeit} & \checkmark\\
	\hline
	\ref{sec:muster_portabilitaet} \nameref{sec:muster_portabilitaet} & \checkmark\\
	\hline
	\ref{sec:muster_effizienz} \nameref{sec:muster_effizienz} & \checkmark\\
	\hline
	\ref{sec:muster_wiederverwendbarkeit} \nameref{sec:muster_wiederverwendbarkeit} & \checkmark\\
	\hline
	\multicolumn{2}{|l|}{\textbf{Webspezifische Kriterien:}} \\
	\hline
	\ref{sec:muster_browser_kompatibilitaet} \nameref{sec:muster_browser_kompatibilitaet} & \checkmark\\
	\hline
	\ref{sec:muster_interoperabilitaet} \nameref{sec:muster_interoperabilitaet} & teilweise erfüllt\\
	\hline
	\ref{sec:muster_asynchrone_verarbeitung} \nameref{sec:muster_asynchrone_verarbeitung} & \checkmark\\
	\hline
	\ref{sec:muster_dateigroesse} \nameref{sec:muster_dateigroesse} & teilweise erfüllt\\
	\hline
\end{tabular}
\caption{Auswertung der Versuchskriterien}
\label{fig:Auswertungstabelle}
\end{figure}

\subsubsection{Auswertung: \ref{sec:muster_wartbarkeit_und_lesbarkeit} \nameref{sec:muster_wartbarkeit_und_lesbarkeit}}
Die Programmiersprache Elm lässt die Erstellung von manuellen Kommentaren zu. Die Kommentare können sich dabei über mehrere Zeilen erstrecken, oder lediglich eine einzige Zeile einnehmen. Für die jeweilige Distanz eines Kommentar gibt es gesonderte Kommandos. Das Atom-Plugin $elm-format$ fügt außerdem Zeilenabstände vor, sowie hinter dem Kommentar ein, so dass es übersichtlich platziert wird. Die Grundlagen dieses Kriteriums wurden somit an dieser Stelle erfüllt. Das fortgeschrittene Kriterium der automatischen Erstellung von Kommentaren mit Informationen über beispielsweise die erwarteten Typen der Übergabeparameter einer Funktion sind nur bedingt erfüllt. Die Möglichkeit $Signaturen$ innerhalb der Elm-Applikation hinzuzufügen besteht und wird durch den $elm-compiler$ unterstützt, allerdings nicht komplett automatisiert. Der Compiler gibt einen Signaturvorschlag in Form einer Warnung ab, sobald die Applikation kompiliert wird, wie es in Abbildung~\ref{fig:compiler-signature-suggestion} demonstriert wird.
\begin{figure}[h]
\centering
\includegraphics[scale=0.52]{img/compiler_sig_suggestion.png}
\caption{Automatisierter Signaturvorschlag durch den $elm-compiler$}\label{fig:compiler-signature-suggestion}
\end{figure}
Dem Signaturvorschlag kann unter der Prämisse übernommen werden, dass sämtliche zuvor definierten Typen korrekt sind, da der $elm-compiler$ die Vorschläge auf der vorherigen Analyse des gesamten Dokumentes stützt.

\subsubsection{Auswertung: \ref{sec:muster_zuverlaessigkeit} \nameref{sec:muster_zuverlaessigkeit}}
Um die Zuverlässigkeit des $Elm-Compiler$ und die Fehlermeldungen bewerten zu können, werden syntaktische und semantische Fehler innerhalb der Applikation eingebaut. Daraufhin wird eine erwartete Fehlermeldung der tatsächlichen Fehlermeldung gegenübergestellt und die Qualität bewertet.

\textbf{1. Entfernen eines importieren Moduls}\\
\begin{figure}[h!]
\centering
\includegraphics[scale=0.5]{img/import-error.png}
\end{figure}
Die Fehlermeldung des 1. Versuches sagt aus, dass die Variable $Navigation.view$ nicht gefunden wurde und das Modul $Navigation$ nicht im aktuellen Namensraum vertreten ist. Ein Entwickler kann daraus schließen, dass das Modul nicht ordnungsgemäß eingebunden wurde, oder im Namensraum anders benannt wurde. In beiden Fällen wird der Entwickler die importierende Programmstelle aufsuchen, die sich am Beginn der Datei befindet. Die Fehlermeldung gibt einen hilfreichen Aufschluss über das eigentliche Problem.


\textbf{2. Falsche Benennung eines Feldes}\\
\begin{figure}[h!]
\centering
\includegraphics[scale=0.4]{img/definition-error.png}
\end{figure}
Im 2. Versuch wird durch die Fehlermeldung vermittelt, dass das $Model$ per Definition ein Feld mit dem Namen $height$ hat. Im weiteren Verlauf der Applikation hingegen wird ebenfalls auf ein $Model$ zugegriffen, diesmal jedoch auf das Feld mit dem Namen $heigth$. Es handelt sich offensichtlich um einen Schreibfehler. Bei der herkömmlichen Entwicklung mit \ac{JS} ist es möglich, dynamisch Felder in einem Objekt hinzuzufügen, wodurch selbst bei einer expliziten Überprüfung des Quellcodes der Fehler erst während der Laufzeit auftreten würde. Die Fehlermeldung weißt hier explizit auf das Problem hin und veranschaulicht es.

\textbf{3. Falsche Typzuweisung eines Feldes}\\
\begin{figure}[h!]
\centering
\includegraphics[scale=0.5]{img/types-error.png}
\end{figure}
In der 3. Fehlermeldung wird durch den Compiler erkannt, dass ein Typenfehler aufgetreten ist. Es ist versucht worden das Feld $height$ mit dem Wert $"0"$ zu initialisieren. Lauf Definition ist das Feld allerdings von Typ $Int$, wohingegen versucht wird einen $String$ zuzuordnen. Da Elm statisch typisiert ist, können Variablen nur einen Datentyp annehmen. \ac{JS} liefert die Möglichkeit eine Variable dynamisch zu typisieren. Der Entwickler wird durch die Fehlermeldung explizit auf den entstandenen Fehler hingewiesen.

\textbf{4. Einbau redundanter Kontrollen}\\
\begin{figure}[h!]
\centering
\includegraphics[scale=0.5]{img/redundant-pattern-code.png}
\includegraphics[scale=0.5]{img/redundant-pattern-error.png}
\end{figure}
Für die Demonstration der 4. Fehlermeldung wird einer Kontrollstruktur ein allumfassender Fall hinzugefügt. Der erste Fall der Konstrollstruktur in der Abbildung~\ref{fig:redundant-pattern} ist für jeden zu überprüfenden Fall gültig. Dementsprechend sind alle nachfolgenden Fälle überflüssig, da sie nie erreicht werden können. Der allumfassende Fall kann mit einem $if(true)$-Statement verglichen werden. Jeglicher $else$-Fall in einem solchen Konstrukt wird unmöglich eintreffen. Der $elm-compiler$ erkennt diese Redundanz und gibt an, dass die Fälle bereits behandelt werden und entfernt werden sollten. Dadurch wird zusätzlich die Lesbarkeit des Quellcodes gewahrt, da der Entwickler aufgefordert wird unnötige Codeteile zu entfernen. Erneut gibt die Fehlermeldung ganz klar an, welche Schritte notwendig sind um den Fehler zu beheben.

\textbf{5. Entfernen einer schließenden Klammer}\\
\begin{figure}[h!]
\centering
\includegraphics[scale=0.5]{img/missing-brackets-error.png}
\end{figure}
Der 5. Versuch zeigt die Fehlermeldung nachdem eine schließende Klammer eines Statements entfernt wurde. Die daraus resultierende Fehlermeldung gibt an, dass entweder Teile innerhalb der Deklaration fehlen, oder Leerzeichen hinzugefügt werden müssen, um die passende Einrückung zu erreichen. Offensichtlich $fehlt$ etwas in der Deklaration, die Fehlermeldung sollte jedoch wesentlich genauer ausfallen und auf die fehlende Klammer hindeuten. Der Entwickler muss nun die dazugehörige Stelle des Quellcodes absuchen, ohne genau zu wissen, welche möglichen Zeichen fehlen. Diese Fehlermeldung ist nicht ausreichend explizit, um den Fehler ohne Probleme ausfindig zu machen.

\textbf{6. Mehrfache Definition einer Funktion}\\
\begin{figure}[h!]
\centering
\includegraphics[scale=0.5]{img/multiple-function-definitions.png}
\end{figure}
Die 6. Fehlermeldung bezieht sich auf die mehrmalige Deklaration einer Funktion. Der $elm-compiler$ erkennt, dass mehrere Funktionen mit demselben Namen existieren und sich im gleichen Namensraum befinden. Der Entwickler wird durch die Fehlermeldung dazu aufgefordert, die Funktionen einzigartig zu benennen. Durch die Fehlermeldung erhält der Entwickler eine klare Problemlösung präsentiert.

Die erzeugten Fehler decken offensichtlich nicht alle möglichen Fehlerfälle ab, geben jedoch recht gut Aufschluss über das Verhalten des $elm-compiler$ hinsichtlich gängiger Fehler. Zusammenfassend lässt sich sagen, dass die Fehlermeldungen des $elm-compiler$ zuverlässig Fehler finden, selbst wenn sie über die syntaktische Ebene hinausgehen und den Entwickler über unnötige Kontrollstrukturen informieren. Ferner wird die Qualität der Fehlermeldungen durch das Markieren der tatsächlichen Fehlerposition und dem Unterstreichen der Fehlerquelle verbessert.
Da alle eingebauten Fehler gefunden und die dazugehörigen Fehlermeldungen im Durchschnitt sehr akkurat auf den Fehler hingewiesen haben, kann das Kriterium der Zuverlässigkeit als erfüllt angesehen werden.


\subsubsection{Auswertung: \ref{sec:muster_portabilitaet} \nameref{sec:muster_portabilitaet}}
Die offizielle Webseite von Elm bietet Installationsdateien und -anleitungen für die gängigen Betriebssysteme $Mac$ und $Windows$, sowie für alle Plattformen die den \ac{NPM} unterstützen. Testweise wurde die Elm-Version der \ac{SPA} unter Ubuntu 14.04 64bit, Windows 7 64bit und Windows 10 64bit kompiliert und auf Warnungen oder Fehler seitens des $elm-compiler$ überprüft. In keinem der Fälle kam es zu Fehlern, geschweige denn Warnungen. Das Verhalten des Compilers war unter allen Betriebssystemen gleich und erzeugte keinerlei Anomalien. Das Betriebssystem $Mac$ konnte nicht getestet werden, da zum Zeitpunkt der Auswertung kein passendes Endgerät zur Verfügung stand. Da sowohl $Mac$, als auch $Linux$ auf dem Betriebssystem $Unix$ basieren \cite{http://www.itwissen.info/definition/lexikon/UNIX.html}, wird an dieser Stelle davon ausgegangen, dass der Kompiliervorgang auch unter $Mac$ ohne signifikante Probleme ausführbar ist. Die Abbildung~\ref{fig:elm-compile} zeigt den fehlerfreien Kompiliervorgang der \ac{SPA} unter Ubuntu 14.04 64bit mit Elm 0.17.
\begin{figure}[h]
\centering
\includegraphics[scale=0.5]{img/elm-compile.png}
\caption{Fehlerfreier Kompiliervorgang durch den Elm-Compiler}\label{fig:elm-compile}
\end{figure}
Das Kriterium der Zuverlässigkeit gilt somit als vollständig erfüllt.


\subsubsection{Auswertung: \ref{sec:muster_effizienz} \nameref{sec:muster_effizienz}}
Innerhalb der Abbildung~\ref{fig:compiler-times} können die Zeiten für die Kompilierung der Elm-Applikation betrachtet werden. Dabei wurde die Dauer des Kompiliervorganges mit und ohne Caching gemessen. Es wurden jeweils zehn Messwerte ermittelt, der jeweils höchste und niedrigste Wert gestrichen und das arithmetische Mittel der verbleibenden acht Werte berechnet. Das Streichen des höchsten und niedrigsten Wert soll etwaigen Messfehlern entgegenwirken. Im Durchschnitt benötigt es 8.03 Sekunden, um die Applikation von Grund auf zu kompilieren. Erlaubt man das Caching der vorherigen Kompiliervorgänge, so ist der erste Durchgang mit 7.88 Sekunden vergleichsweise hoch. Die folgenden Durchgänge hingegen benötigen im Durchschnitt 0.3 Sekunden, unter der Berücksichtigung, dass auch hier der höchste und niedrigste Wert vernachlässigt wird. Dadurch fällt der initial aufwendigste Kompiliervorgang weg. Unter realen Umständen wird ein Entwickler die Applikation unter der Verwendung von Caching kompilieren, womit nur sehr wenig Zeit verloren geht. Ferner kann der Entwickler auf den mitgelieferten Webserver $elm-reactor$ zurückgreifen, der ebenfalls die Applikation bei jedem Seitenaufruf, jedoch nur die tatsächlichen Änderungen neu kompiliert. In beiden Fällen sind die Werte von 8.03 beziehungsweise 0.3 Sekunden für das Kompilieren mehr als annehmbar, insbesondere da ein Entwickler standardmäßig das Caching nutzt. Die Effizienz wird als erfüllt angesehen.
\begin{figure}[h]
\centering
\begin{tabular}{ | c | c | c |}
	\hline
	 \textbf{Durchlauf} 				& \textbf{Dauer ohne Caching} 	& \textbf{Dauer mit Caching}\\
	 \hline
	 1 & 7,56s & \st{7,88s} \\
	 \hline
	 2 & 8,31s & 0,34s\\
	 \hline
	 3 & 7,78s & 0,34s\\
	 \hline
	 4 & 7,77s & 0,31s\\
	 \hline
	 5 & \st{7,4s }& 0,22s\\
	 \hline
	 6 & 8,48s & 0,28s\\
	 \hline
	 7 & \st{9,79s} & \st{0,19s}\\
	 \hline
	 8 & 9,19s & 0,30s\\
	 \hline
	 9 & 6,98s & 0,33s\\
	 \hline
	 10 & 8,14s & 0,28s\\
	 \hhline{|=|=|=|}
	 \textbf{Durchschnitt:} & \textbf{8,03s} & \textbf{0,3s}\\
	 \hline
\end{tabular}
\caption{Zeitliche Auswertung des Kompiliervorganges in Elm}\label{fig:compiler-times}
\end{figure}
Die Performanz der Programmiersprache Elm wird mittels der Applikation $TodoMVC\,Performance\,Comparison$ ausgewertet. Die Applikation liefert Ergebnisse für die Benchmark-Tests der Programmiersprachen Backbone 1.1.2, Ember 1.4.0 mit Handlebars 1.30, Angular 1.2.14, React 0.10.0, Om 0.5.0 mit React 0.9.0, Mercury 3.1.7 mit virtual-dom 0.8, sowie Elm 0.12.3 mit virtual-dom 0.8. Dabei nutzen die Programmiersprachen Elm, Om und Mercury das Konzept der $virtual-dom$.
\begin{figure}[ht]
\centering
\includegraphics[scale=0.6]{img/elm-benchmark-final.png}
\caption{Benchmark der TodoMVC in unterschiedlichen Programmiersprachen}\label{fig:elm-todo-benchmarks}
\end{figure}
Die zu testenden Programmiersprachen sind weit verbreitet in der Webentwicklung. Die Abbildung~\ref{fig:elm-todo-benchmarks} zeigt eine graphische Auswertung nach 20 ausgeführten Tests. Bei jedem Testdurchlauf wird dabei die Zeit gemessen, die für die Erstellung von 100 Elementen, dem Anklicken all dieser erstellten Elemente sowie dem Löschen aller Elemente benötigt wird. Der Test wurde auf dem in der Sektion~\ref{sec:Entwicklungsumgebung} beschriebenem System ausgeführt. Abbildung~\ref{fig:elm-todo-benchmarks} zeigt, dass Elm durchschnittlich 259ms für einen Test benötigt hat. Den niedrigsten Wert mit 191ms pro Durchlauf erzielte Mercury und ist damit mehr als ein viertel mal schneller als Elm. Den schlechtesten Wert erzielte das Framework AngularJs mit 1832ms pro Durchlauf. Dieser Wert schneidet mehr als 6x schlechter ab, als es bei Elm der Fall war.
Obwohl die gegenübergestellten Programmiersprachen in der Testumgebung nicht den neuesten Versionen entsprechen, ist davon auszugehen, dass die grobe Platzierung selbst mit den neuesten Versionen ähnlich ausfällt. Die einzelnen Frameworks haben an der grundlegenden Umsetzung zur Darstellung von Daten im Browser nichts geändert, das bedeutet, dass Elm, Mercury und Om weiterhin mit neueren Versionen der $virtual-dom$ arbeiten. Die Frameworks Angular, Backbone und Ember nutzen noch immer nicht standardmäßig das Konzept der $virtual-dom$. Lediglich das Framework React könnte bei einem Test der neuesten Version signifikante Veränderung aufweisen, da hier ein $virtual-dom$ Konzept implementiert wurde.
Das Kriterium der Performanz einer Elm-Applikation gilt als erfüllt. Im Vergleich zu anderen beliebten Frameworks schlägt sich Elm mit Bravour und wird nur von $Mercury$, einem extrem leichtgewichtigem Framework übertrumpft.

\subsubsection{Auswertung: \ref{sec:muster_wiederverwendbarkeit} \nameref{sec:muster_wiederverwendbarkeit}}
Das Konzept der Wiederverwendbarkeit, insbesondere der Modularität wird in Elm komplett durchgesetzt. Jede Datei in Elm ist zwangsläufig ein Modul, insofern man die darin enthaltenen Funktionen nutzen möchte. Zusätzlich ist auch ein Zugriffsschutz gegeben. Das bedeutet, dass Funktionen gesondert nach außen hin sichtbar und somit nutzbar gemacht werden können. Gängige imperative Programmiersprachen wie beispielsweise $C++$ nutzen Klassen anstelle von Modulen. Dabei wird der Zugriff von außen auf eine Funktion innerhalb der Klasse durch die Schlüsselwörter $public$, $private$ und $protected$ definiert. Jedes Schlüsselwort bietet einen sichereren Zugriffsschutz. Ähnlich fungiert in Elm das Schlüsselwort $exposing$, wodurch einzelne Funktionen an die importierenden Module zugreifbar gemacht werden. Der Entwickler hat dadurch die Sicherheit, dass nur die von ihm erwarteten Funktionen, sprich die \ac{API}, genutzt werden. Dementsprechend ist das Bewertungskriterium der Wiederverwendbarkeit in allen Aspekten erfüllt.


\subsubsection{Auswertung: \ref{sec:muster_browser_kompatibilitaet} \nameref{sec:muster_browser_kompatibilitaet}}
Wie in Sektion~\ref{sec:Entwicklungsumgebung} angesprochen, werden lediglich die Browser Google Chrome (Version 51), Internet Explorer (Version XY), Mozilla Firefox (Version 47) und Opera (Version 38) getestet. Dabei wird in jedem Browser die $index.html$ der \ac{SPA} aufgerufen und gewartet, bis die Seite komplett geladen ist. Anschließend werden die $onClick$-Elemente der Applikation getestet, indem auf die einzelnen Reiter in der Navigation geklickt wird. Die Webseite soll daraufhin zu den entsprechenden Stellen im Dokument scrollen und die Navigationsleiste kleiner werden. Sobald die $Portfolio$-Sektion erreicht wurde, wird das erste Element angeklickt und überprüft, ob ein asynchroner Request erzeugt, sowie die Ergebnisse in das geöffnete $Modal$ eingepflanzt wurden.

Die Darstellung in allen Browsern ist komplett gleich. Es gibt nur marginale Unterschiede in der Darstellung der \ac{SPA} selbst, die jedoch auf die Unterschiede der Browser selbst zurückzuführen sind. Zusätzlich erzeugt der Aufruf und die Interaktion mit der Webseite in keinem getesteten Browser einen Fehler. Ferner funktioniert die asynchrone Verarbeitung nach einem Klick auf eines der Portfolio-Elemente problemlos und zügig. Auch die Darstellung des Modals unterscheidet sich nicht. Alles in Allem scheint sich die kompilierte Elm-Applikation in allen getesteten Browsern gleich zu verhalten. Das Kriterium der Browser-Kompatibilität gilt als komplett erfüllt.

\subsubsection{Auswertung: \ref{sec:muster_interoperabilitaet} \nameref{sec:muster_interoperabilitaet}}
Das Einbinden von externen \ac{JS}-Frameworks oder bestehendem \ac{CSS}-Code ist nicht vollständig in Elm gegeben. Die Programmiersprache Elm liefert mehrere Möglichkeiten eine Elm-Applikation im Browser darzustellen. Die Methoden dafür können bei Bedarf in der Sektion~\nameref{sec:Grundaufbau} nachgelesen werden. Abhängig von der Art der Initialisierung der Elm-Applikation entsteht eine Abhängigkeit zu einer $.html$-Datei, in welcher der Grundaufbau der Webseite erstellt und die Elm-Applikation geladen wird. Nutzt man eine $.html$-Datei und lädt dort die Elm-Applikation hinein, können bestehende \ac{CSS}- oder \ac{JS}-Frameworks innerhalb des $head$-Tag in einem $link$- oder $script$-Tag geladen werden. Hier entstehen keinerlei Probleme. Wird allerdings die Abhängigkeit der $.html$-Datei umgangen, wird die Elm-Applikation entweder im Vollbildmodus, oder im Hintergrund gestartet und muss somit zwangsweise das Einbinden externer Quellen übernehmen. Diese Funktionalität ist gegeben, jedoch sehr umständlich und nicht nutzbar für ein fertiges Produkt. Beim Einbinden von externem \ac{CSS} durch nativen Elm-Code wird ein $link$-Tag innerhalb des $body$-Tags erstellt und der Inhalt asynchron nachgeladen. Üblicherweise wird externer \ac{CSS}-Code innerhalb des $head$-Tags geladen und auf das gesamte Dokument angewandt. Insofern wäre es nicht problematisch, jedoch bewirkt das Einbinden des Stylings innerhalb des $body$-Tags ein flickern, wodurch zunächst der gesamte Inhalt des Dokumentes komplett ohne und kurze Zeit später mit dem definierten \ac{CSS}-Code angezeigt wird. Des Weiteren bietet die $elm-lang/html$-Bibliothek keine native Funktion für einen $script$ oder $link$-Tag. Stattdessen muss die $Html.node$-Funktion genutzt werden, um ein eigenes Html-Element zu erzeugen. Eine Problemlösung steht an dieser Stelle aus.

Abgesehen vom initialen Einbinden externer Quellen funktioniert die Kommunikation zwischen der Elm-Applikation und externen \ac{JS}-Skripten problemlos. Über spezifische Ports die jeweils für nur eine Aufgabe genutzt werden können ist die Kommunikation möglich. Die Änderungen um einen Port zu definieren und ein Modul für diese Aufgabe zu befähigen sind minimal. Der Zustand der Elm-Applikation ist dabei immer gewahrt und unmöglich in einen undefinierten Zustand zu bringen, da vorab definiert wird, welche Typen gesendet oder entgegengenommen werden. Wird beispielsweise ein $String$ erwartet, jedoch ein $Objekt$ seitens einem \ac{JS}-Skript an die Elm-Applikation gesendet, nimmt die Applikation die Daten nicht entgegen und verwirft sie. Handelt es sich hingegen tatsächlich um den erwarteten Typ, sprich einem $String$, werden die Daten an die entsprechende Stelle zur $update$-Funktion weitergeleitet und die Daten ausgewertet.

Das Kriterium der Interoperabilität kann teilweise als erfüllt angesehen werden.
Da die grundlegenden Funktionen des Kriteriums gegeben sind, jedoch nicht unter allen Umständen perfekt funktionieren, ist hier nicht von einer vollständigen Erfüllung auszugehen. Es ist nicht zu erwarten, dass die genannten Probleme bei Weiterführung des aktuell angewandten Konzepts der Programmiersprache behebbar sind. Die Elm-Applikation wird asynchron als eigenständiges Element innerhalb der \ac{DOM} initialisiert. Dadurch ist es theoretisch notwendig zunächst die Applikation vollständig zu laden und im Nachhinein Änderungen am $head$-Tag, sprich außerhalb der Elm-Applikation, zu erlauben, um so weitere \ac{CSS} oder \ac{JS}-Dateien laden zu können. Erst wenn diese Vorgänge abgeschlossen sind, sollte dem Nutzer das Ergebnis der Webseite präsentiert werden. Folgt man dieser Herangehensweise würde dies zu weiteren offensichtlichen Problemen führen, allen voran die initiale Ladezeit, bis der Nutzer die Webseite fertig geladen betrachten kann. Alles in Allem bedarf es für eine Problemlösung weiterer Recherchen und voraussichtlich die Umgestaltung der in Elm genutzten Konzepte.

\subsubsection{Auswertung: \ref{sec:muster_asynchrone_verarbeitung} \nameref{sec:muster_asynchrone_verarbeitung}}
Asynchrone Verarbeitung innerhalb einer nativen Elm-Applikation ist problemlos möglich. Dabei macht es keinen Unterschied, ob die asynchrone Aufgabe eine langwierige Berechnung, oder das Anfordern von externen Daten mittels eines \ac{HTTP}-Requests ist. Die Bibliothek $elm-lang/core$ liefert zur asynchronen Verarbeitung den Typ $Tasks$. Mit Hilfe dieses Typs  können asynchrone Aufgaben definiert werden, die durch interne Algorithmen asynchron ausgeführt werden. Besteht bei einer Aufgabe die Gefahr, dass sie fehlschlagen könnte, können die Funktionen $fail$ und $succeed$ zur Hilfe genommen werden. So kann der Zustand der Elm-Applikation in jedem Falle gesichert werden. Innerhalb der Applikation wurde ein asynchroner Request ausgeführt, bei dem von einem externen Webserver Daten angefordert wurden. Dadurch entstand die eine Abhängigkeit gegenüber der Antwortzeit des Webservers. Diese war unbekannt und konnte nicht vorhergesehen werden. Trotzdessen war die Applikation auch während der Anfrage in vollem Umfang nutzbar. Sobald die Antwort des Webservers von der Elm-Applikation entgegengenommen werden konnte, wurden die übertragenen Daten dekodiert und in das $model$ eingearbeitet. Daraufhin wurden die übermittelten Daten für den $view$ verfügbar gemacht und eine aktualisierte Version der gesamten Applikation auf dem Bildschirm ausgegeben.
Versuchsweise wurde die Übertragungszeit künstlich erhöht, so dass mehrere Sekunden auf die Antwort des Servers gewartet wurde. Dank der Umsetzung der asynchronen Verarbeitung kam es jedoch weder zu einem Laufzeitfehler, noch einer unvorhergesehenen Darstellung von Information. Ein weiterer Grund für die erwartete Darstellung ist, dass das Feld $async\_content$ in das die asynchronen Daten gespeichert werden, bereits bei der Initialisierung des $model$ mit einem initialen Wert, genauer einem leeren String, versehen wurde. Dementsprechend wurde an der besagten Stelle in der das Feld des $model$ dargestellt werden sollte ein leerer String angezeigt. Bei der Simulation einer fehlerhaften Übertragung wurde ebenfalls der Initialwert des Feldes angezeigt. Sämtliche Fälle die eintreten konnten wurden getestet und hatte in keinem Fall einen undefinierten Zustand der Elm-Applikation zur Folge.
Das Kriterium kann somit als vollständig erfüllt angesehen werden.

\subsubsection{Auswertung: \ref{sec:muster_dateigroesse} \nameref{sec:muster_dateigroesse}}
Ermittelt wurden die Dateigrößen der Elm-Applikation, die in Abbildung~\ref{fig:dateigroesse-stats} zusammengefasst wurden, indem eine neue Elm-Applikation initialisiert wurde. Dafür wurden die Packete $elm-lang/html$ (Version 1.0.0), $elm-lang/core$ (Version 4.0.1) und $elm-lang/virtual-dom$ (Version 1.0.2) installiert und in einem Testmodul $Main.elm$ importiert.
Das Testmodul beinhaltete zusätzlich die Funktionen $model$, $view$, $update$ und $main$, um dem \ac{MVU}-Konzept von Elm gerecht zu werden und einen minimal notwendigen Grundaufbau für eine funktionierende Applikation zu simulieren. Dabei liefert $model$ einen Integer $0$ zurück. Die Funktion $view$ gibt den Wert von $model$ in einem \ac{HTML}-Element aus und die $update$-Funktion nimmt sämtliche Interaktionen des Nutzers entgegen und gibt das unveränderte $model$ zurück. Die $main$-Funktion macht dabei die Elm-Applikation lauffähig über den Webserver $elm-reactor$ oder einer expliziten Einbindung der Applikation in eine $.html$-Datei. 
\begin{figure}[h!]
	\begin{tabular}{ | p{7cm} | c | c |}
	\hline
	 \textbf{Name} 				& \textbf{Minified} 	& \textbf{Minified + gzipped}\\
	 \hline
	 Angular2 					& 566KB	& 111KB\\
	 \hline
	 Ember 2.2.0 				& 435KB & 111KB\\
	 \hline
	 React 0.14.5 + React DOM	& 133KB & 40KB\\
	 \hline
	 Elm 0.17					& 127KB & 34KB\\
	 \hline
	\end{tabular}
\caption{Dateigrößen der Frameworks Angular2, Elm, Ember und React in Kilobyte}\label{fig:dateigroesse-stats}
\end{figure}
Das Resultat der Kompilierung der Elm-Applikation misst 127 Kilobyte. Der $elm-compiler$ liefert keinerlei Möglichkeit die Dateigröße zu minimieren. Aufgrunddessen wird das externe Tool $jscompress$, verfügbar auf der Webseite \url{http://jscompress.com/}, genutzt um die Dateigröße der kompilierten Elm-Applikation zu minimieren. Das Resultat ist eine Minimierung um 66\%. Die Dateigröße misst nun 59 Kilobyte. Führt man nun noch das gzip-Verfahren auf die minimierte Datei aus, misst die daraus resultierende Datei noch 34 Kilobyte. Im Gegensatz zu den Dateigrößen der Frameworks $AngularJs$, $Ember$ und $React$ ist Elm um bis zu XY\% leichtgewichtiger (ausgehend von den minimierten und mit GZip komprimierten Datei) und erfüllt somit das Kriterium einer geringen Dateigröße. Jedoch hat bietet der $elm-compiler$ keinerlei Möglichkeit die Minimierung, in Form von $gzip$, dem Entfernen von Leerzeichen oder der Verkürzung von Variablennamen, durchzuführen. Aufgrund dieser Tatsache gilt das Kriterium als nur teilweise erfüllt.


%Minified:
%ls -lhS
%566K Jan  4 22:03 angular2.min.js
%563K Jan  4 22:05 angular2.0.0-beta.0-all.umd.min.js
%486K Jan  4 21:50 ember.1.13.8.min.js
%435K Jan  4 21:48 ember.2.2.0.min.js
%205K Jan  4 22:06 angular2.0.0-beta.0-Rx.min.js
%144K Jan  4 21:59 react-with-addons-0.14.5.min.js
%143K Jan  4 21:46 angular.1.4.5.min.js
%132K Jan  4 21:56 react-0.14.5.min.js
%121K Jan  4 21:35 angular.1.3.2.min.js
%5.3K Jan  4 22:00 redux-3.0.5.min.js
%706B Jan  4 21:57 react-dom-0.14.5.min.js


%GZipped:
%$ gzip -r .
%$ ls -lhS
%123K Jan  4 22:11 ember.1.13.8.min.js.gz
%119K Jan  4 22:11 angular2.0.0-beta.0-all.umd.min.js.gz
%111K Jan  4 22:11 ember.2.2.0.min.js.gz
%111K Jan  4 22:11 angular2.min.js.gz
%51K Jan  4 22:11 angular.1.4.5.min.js.gz
%45K Jan  4 22:11 angular.1.3.2.min.js.gz
%42K Jan  4 22:11 react-with-addons-0.14.5.min.js.gz
%39K Jan  4 22:11 react-0.14.5.min.js.gz
%32K Jan  4 22:11 angular2.0.0-beta.0-Rx.min.js.gz
%1.9K Jan  4 22:11 redux-3.0.5.min.js.gz
%455B Jan  4 22:11 react-dom-0.14.5.min.js.gz


\chapter{Fazit}
\label{chap:fazit}
Ziel dieser wissenschaftlichen Arbeit war es, die Programmiersprache Elm auf unterschiedlichste Gesichtspunkte hin zu evaluieren und speziell für die Verwendung im Bereich der Webentwicklung zu prüfen. Infolgedessen wurde eine \ac{SPA} überführt in eine native Elm-Applikation. In der herkömmlichen Webentwicklung ist es üblich die darstellbare Oberfläche einer Webseite mit \ac{HTML}, das Styling mit \ac{CSS} und Interaktionen mit \ac{JS} zu programmieren. Ein Entwickler hat dadurch zwangsweise drei Programmiersprachen zu bedienen. Einerseits mündet dieses Konzept in einer klaren Trennung der Applikation, andererseits muss jede Programmiersprache dem Entwickler bekannt sein. Ferner werden \ac{JS}-Skripte genutzt, um die dargestellte Webseite weiter zu verändern, Interaktionen des Nutzers abzufangen und darauf zu reagieren, wodurch die Webseite unter Umständen undefinierte Zustände erreicht und ausgeführte Skripte im Konflikt miteinander stehen.
Elm hingegen bringt die funktionale Programmierung mit einem expliziten Typensystem, unveränderbaren Variablen und puren Funktionen, sowie klar strukturierten Style-Guides und den Ansprüchen eines Entwicklers im Hinterkopf in die Webentwicklung. Das hinter Elm stehende Konzept beruht darauf, zuverlässige Applikationen zu entwickeln. Der initiale Aufwand bei der Entwicklung einer Elm-Applikation ist dabei höher als bei der üblichen Entwicklung von Webseiten, der Aufwand die Applikation zu erweitern, Codeteile auszulagern und ungenutzte Teile zu entfernen ist hingegen wesentlich unkomplizierter.
Zu Beginn der praktischen Ausarbeitung war unklar, inwiefern die native Implementierung der \ac{SPA} in Elm mit den bestehenden \ac{JS}-Skripten und \ac{CSS}-Definitionen funktionieren wird und ob etwaige Änderungen notwendig sind. Bestenfalls wären sämtliche Garantien die Elm liefert, mit der Überführung des Views in der Applikation eingebunden, ohne die \ac{JS}- und \ac{CSS}-Dateien verändern zu müssen.
Die Überführung der Elm-Applikation ergab, dass lediglich zwei der insgesamt neun geprüften Kriterien nur teilweise erfüllt wurde. Die verbleibenden Kriterien wurden vollständig erfüllt. Zu den unerfüllten Kriterien gehörte unter anderem die Interoperabilität, mit der geklärt werden sollte, inwiefern die Funktionalität vorhandener \ac{JS}-Skripte gegeben sei, wenn der Rest der Applikation nativ in Elm verwirklicht wird. Ergebnis der Auswertung war, dass nicht alle \ac{JS}-Skripte einwandfrei mit dem nativ in Elm programmierten Teil der Applikation, genauer der \ac{DOM}, kommunizierten. Die Ereignisbehandler, die durch die bestehenden \ac{JS}-Skripte erzeugt wurden, bezogen sich dabei auf Elemente die zur Zeit ihrer Initialisierung noch nicht im Zugriffsbereich des Skriptes befanden, da die Elm-Applikation erst nachträglich in die \ac{DOM} injiziert wurde. Zusätzlich nutzt Elm das Konzept der $virtual-dom$, wodurch die tatsächliche Repräsentation des \ac{DOM} nicht direkt von außen zugreifbar ist und sich unter Umständen verändert, wodurch die vorherigen Ereignisbehandler nicht mehr auf die ursprünglichen Elemente in der \ac{DOM} zugreifen. Die Dateigröße einer Elm-Applikation mit den Grundfunktionen zur Darstellung eines \ac{HTML}-Elementes im Browser war zufriedenstellend gering. Jedoch wird die erzeugte Datei nicht minimiert, obwohl dies mit einfachsten Mitteln bewerkstelligt werden kann. Der Test ergab, dass die Datei um mehr als 60\% kleiner sein könnte, wenn eine standardmäßige Verkleinerung der Datei durch den $elm-compiler$ stattfinden würde.

Der selbst erzeugte Quellcode wies eine sehr hohe Wartbarkeit und Lesbarkeit auf, zum einen wegen der Vielzahl an unterstützenden Plugins die es für die Entwicklungsumgebungen gibt, zum anderen aufgrund der Hilfe durch den $elm-compiler$. Hier wird dem Entwickler viel Arbeit erspart und gleichzeitig Sicherheit über die Richtigkeit der Applikation geliefert. Die Fehlermeldungen waren sehr detailliert und überwiegend akkurat, wodurch eine Elm-Applikation, sollte sie fehlerfrei kompiliert werden können, als zuverlässig angesehen werden kann. Zuverlässig ist der erzeugte Quellcode auch hinsichtlich der Nutzung auf unterschiedlichen Platformen. Es gab keinerlei Fehler bei der Kompilierung des Codes unter den gängigsten Betriebssystemen. Trotz der Plattformunabhängigkeit überzeugte die Elm-Applikation mit hoher Effizienz. Die getestete Applikation erwies sich als deutlich schneller als andere beliebte Frameworks wie AngularJS oder React. Obwohl sich die Benchmarks auf ältere Versionen der getesteten Frameworks stützen, ist davon auszugehen, dass die Ergebnisse der aktuellsten Versionen ähnlich ausfallen würden. Um exakte Ergebnisse zu ermitteln, müsste die Applikation $TodoMVC\,Performance\,Comparison$ in allen Programmiersprachen auf die neueste Version aktualisiert werden.
Ferner war es denkbar einfach vorhandenen Code in eigene Funktionen und Module auszulagern, um eine bessere Übersichtlichkeit innerhalb des Quellcodes zu erreichen. Dabei können Module beliebig oft an anderen Stellen eingebunden und genutzt werden.
Die Elm-Applikation kompilierte nicht nur auf allen getesteten Plattformen, sondern bot auch eine erstaunliche Kompatibilität mit einer Vielzahl von Browsern. Die Darstellung in den getesteten Browsern war stets fehlerfrei und erzeugte fast gänzlich gleiche Ergebnisse, die nur marginale Unterschiede aufwiesen.
Trotz der unveränderbaren Datenstrukturen in Elm war es möglich Daten asynchron zu verarbeiten. Mit Hilfe eines $Tasks$ konnte ein asynchroner Request an einen externen Server gesendet, die Antwort ausgewertet und auf dem Bildschirm dargestellt werden, ohne den Zustand der Applikation zu gefährden und einen Absturz zu erzeugen.

Zusammenfassend lässt sich sagen, dass Elm die Anforderungen der Webentwicklung fast vollständig erfüllt. Lediglich die Interoperabilität bedarf der Umsetzung eines anderen Konzeptes.
Sämtliche Tests dieser wissenschaftlichen Arbeit beziehen sich bewusst auf grundlegende Aspekte der Programmiersprache, um so allgemeine Erkenntnisse daraus ziehen zu können. Die \ac{SPA} ist denkbar einfach aufgebaut, zeigt jedoch alle Konzepte die Elm nutzt und für wichtig erklärt. Während der Entwicklung der Elm-Applikation wurde deutlich, dass das $Model$ den Status der Applikation innehält. Dadurch schien es, als ob der Fokus während der Entwicklung eher auf den zu verarbeitenden Daten, als auf der letztlichen Darstellung lag. Die Programmiersprache Elm bietet sich folglich eher für Web-Applikationen an, die deutlich mehr Logik als eine \ac{SPA} beinhalten, oder mehr Interaktionen mit dem Nutzer anstreben.

%\begin{enumerate}
%	\item{Gegenüberstellung}
%		\subitem{-} Was war das Ziel und der Wissensstand zu Beginn
%		\subitem{-} Was waren die Erwartungen
%		\subitem{-} Gab es unvorhergesehene Probleme? Wenn ja, welche?
%		\subitem{-} Entsprach das Endergebnis (und der Weg dorthin) dem vorherigen Ziel?
%	\item Aufzählung der maßgeblich wichtigen Punkte
%	\item Résumé
%		\subitem{-} Wie verallgemeinerbar sind die gesammelten Daten?
%		\subitem{-} Erfüllt Elm die Anforderungen der Webentwicklung?
%		\subitem{-} Einstufung, für welche Anwender Elm geeignet ist
%		\subitem{-} Aktuell (noch) bestehende Probleme, die es zu lösen gilt
%\end{enumerate}





%Literaturverzeichnis
\clearpage % or \cleardoublepage
\phantomsection
\addcontentsline{toc}{chapter}{Literaturverzeichnis}
\bibliographystyle{plain}
\bibliography{Literatur}

\clearpage % or \cleardoublepage
\phantomsection
\addcontentsline{toc}{chapter}{Eidesstattliche Erklärung}
\chapter{Eidesstattliche Erklärung}
\label{sec:erklaerung}

Hiermit versichere ich, dass ich die vorliegende Arbeit selbstständig verfasst und ohne Benutzung
anderer als der angegebenen Hilfsmittel angefertigt, nur die angegebenen Quellen benutzt und die
in den benutzten Quellen wörtlich oder inhaltlich entnommenen Stellen als solche kenntlich
gemacht habe.
Die Arbeit hat in gleicher oder ähnlicher Form noch keiner anderen Prüfungsbehörde vorgelegen.\\
\\[1.5cm]
Kiel, den 07. Juli 2016
\\[2cm]
---------------------------------------------
\\
\enspace (Unterschrift)

\iffalse
\newpage
\addsec{Danksagungen}
\label{danksagungen}
Weit hinten, hinter den Wortbergen, fern der Länder Vokalien und Konsonantien leben die Blindtexte. Abgeschieden wohnen Sie in Buchstabhausen an der Küste des Semantik, eines großen Sprachozeans. Ein kleines Bächlein namens Duden fließt durch ihren Ort und versorgt sie mit den nötigen Regelialien. Es ist ein paradiesmatisches Land, in dem einem gebratene Satzteile in den Mund fliegen. Nicht einmal von der allmächtigen Interpunktion werden die Blindtexte beherrscht – ein geradezu unorthographisches Leben. Eines Tages aber beschloß eine kleine Zeile Blindtext, ihr Name war Lorem Ipsum, hinaus zu gehen in die weite Grammatik. Der große Oxmox riet ihr davon ab, da.
\fi

\end{document}
