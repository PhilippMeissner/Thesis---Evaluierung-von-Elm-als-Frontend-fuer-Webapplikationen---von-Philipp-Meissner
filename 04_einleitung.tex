\chapter{Einleitung}
\label{sec:einleitung}
Die Entwicklung von Webapplikationen kostet Zeit und dementsprechend Geld. Doch mit der Veröffentlichung der Webseite ist es nicht getan. Oftmals werden Fehler in der Webapplikation erst zur Laufzeit bemerkt und müssen daraufhin behoben werden. Wieder entstehen Kosten, nicht nur durch den Zeitaufwand der erneuten Entwicklung, sondern auch durch mögliche Ausfälle der Webapplikation. Die neue funktionale Programmiersprache Elm verspricht Besserung, indem Laufzeitfehler der Vergangenheit angehören sollen \cite[Vgl. f.]{elm-no-runtime-errors}. Dies soll nicht zuletzt an dem eigens entwickelten Compiler liegen, der den geschriebenen Quellcode überprüft und eine immense Anzahl von Fehlern finden soll. Zusätzlich sollen die Fehlermeldungen sehr präzise und aussagekräftig sein, wodurch dem Entwickler bei der Programmierung zusätzlich Zeit erspart wird. Darüber hinaus gilt die dargestellte Applikation als sehr performant.
Im Rahmen dieser wissenschaftlichen Arbeit sollen diese Aussagen überprüft werden. Zu diesem Zweck werden in Kapitel \ref{sec:grundlagen} \glqq\nameref{sec:grundlagen}\grqq~notwendige Grundlagen geklärt. Unter anderem behandelt das Kapitel das Konzept funktionaler Programmiersprachen, gefolgt von einer Einführung in die Grundlagen der Programmiersprache Elm. Hier werden  Im Anschluss werden in Kapitel \ref{sec:Evaluierung der Programmiersprache Elm} \glqq\nameref{sec:Evaluierung der Programmiersprache Elm}\grqq~Bewertungskriterien vorgestellt und erläutert, anhand derer die vorherigen Aussagen überprüft werden. Dazu gehören unter anderem die Evaluation der Performanz, die Klärung der Frage wie hilfreich der Elm-Compiler tatsächlich ist und ob mit der Verwendung von Elm Kompromisse bei der Kompatibilität mit unterschiedlichen Browsern gemacht werden müssen. Dem folgt die Dokumentation und Erläuterung des praktischen Teils dieser wissenschaftlichen Arbeit, in der zunächst einmal der geplante Programmablauf, sowie die genutzte Entwicklungsumgebung vorgestellt werden.  und mündet in einer Auswertung anhand der aufgestellten Bewertungskriterien. Auf Grundlage der Auswertung werden im letzten Kapitel \ref{chap:fazit} \glqq\nameref{chap:fazit}\grqq~die Ergebnisse gegenübergestellt. Das Ziel dieser wissenschaftlichen Arbeit soll es sein, die Daseinsberechtigung der Programmiersprache Elm im Bereich der Frontend\footnote{Bezeichnet weitestgehend den Teil einer Webapplikation, der dem Nutzer dargestellt wird.}-Webentwicklung zu prüfen und eine Aussage darüber zu treffen, ob Elm die herkömmliche Entwicklung von Webapplikationen verbessert und tatsächlich eine valide Alternative darstellt.

%Die heutige Welt ist sehr schnelllebig und spielt sich immer mehr im Internet ab. Firmen präsentieren sich auf ihren Webseiten und akquirieren dadurch Neukunden. Ganze Geschäfte leben nur noch durch den Online-Handel und besitzen keinerlei Verkaufsläden, in denen ein Kunde das Produkt vorab in den Händen halten kann. Umso wichtiger ist es, das Produkt auf der Webseite außergewöhnlich gut zu präsentieren, um den Kunden zu überzeugen. Webseiten dieser und im Grunde jeglicher Art zielen immer darauf ab, einem Nutzer Informationen bereitzustellen und entsprechend angenehm zu präsentieren. Doch die Entwicklung solcher Systeme ist komplex und geht weit über das Design hinaus. Folglich ist ein großer Teil der Ausgaben von Firmen, die sich online präsentieren, die Bezahlung von Entwicklern für ihre Webapplikationen. Dies können simple Webseiten ohne großartige Funktionen sein, jedoch auch komplexe Systeme wie ein automatisierter Online-Handel, in dem die Nutzer ihren gesamten Einkauf abwickeln können, mitsamt Bezahlung. Demzufolge ist es für die Firmen von großem Interesse, dass die angestellten Entwickler zügig Ergebnisse in der Entwicklung der Webapplikationen machen. Damit der Entwickler effektiv arbeiten kann, braucht er Systeme, die ihn unterstützen. Angefangen bei den Werkzeugen wie seiner Entwicklungsumgebung, bis hin zur tatsächlichen Programmiersprache. Diese wissenschaftliche Arbeit befasst sich mit der neuen Programmiersprache Elm, welche eine funktionale, den deklarativen Programmierparadigmen folgende, Programmiersprache ist. Sie wurde zu Beginn ihrer Entwicklung für die Erstellung von grafischen Benutzeroberflächen und der Verbildlichung mathematischer Funktionen genutzt, bewegt sich nun jedoch immer weiter in Richtung der Webentwicklung und kommt mit einigen Neuheiten, Veränderungen und einer aktiven Gemeinschaft an Open-Source Entwicklern.
%
%Elm verspricht eine blitzschnelle Darstellung von Inhalten selbst bei riesigen Datenmengen mit Hilfe einer Technik die $virtual-dom$ genannt wird. Auch soll es keinerlei Laufzeitfehler mehr geben, da die gesamte Sprache mit Garantien ausgeschmückt ist, die im Zusammenspiel mit dem eigens entwickelten Compiler alle möglichen Fehler vorab findet und darauf hinweist.
%All diese Versprechungen werden anhand mehrerer Bewertungskriterien während einer empirischen Entwicklung einer Webapplikation geprüft.