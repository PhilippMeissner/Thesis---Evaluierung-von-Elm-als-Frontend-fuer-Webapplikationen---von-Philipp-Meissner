\chapter{Einleitung}
\label{sec:einleitung}
Die Entwicklung von Webapplikationen kostet Zeit und dementsprechend Geld. Doch mit der Veröffentlichung der Webseite ist es nicht getan. Oftmals werden Fehler in der Webapplikation erst zur Laufzeit bemerkt und müssen daraufhin behoben werden. Wieder entstehen Kosten, nicht nur durch den Zeitaufwand der erneuten Entwicklung, sondern auch durch mögliche Ausfälle der Webapplikation hervorgerufen werden. Die neue funktionale Programmiersprache Elm verspricht Besserung, indem Laufzeitfehler der Vergangenheit angehören sollen \cite[Vgl. f.]{elm-no-runtime-errors}. Dies soll nicht zuletzt an dem eigens entwickelten Compiler liegen, der den geschriebenen Quellcode überprüft und eine immense Anzahl von Fehlern finden soll. Zusätzlich sollen die Fehlermeldungen sehr präzise und aussagekräftig sein, wodurch dem Entwickler bei der Programmierung zusätzlich Zeit erspart wird. Darüber hinaus gilt die dargestellte Applikation als sehr performant.
Im Rahmen dieser wissenschaftlichen Arbeit sollen diese Aussagen überprüft werden. Zu diesem Zweck werden in Kapitel \ref{sec:grundlagen} \glqq\nameref{sec:grundlagen}\grqq~notwendige Grundlagen geklärt. Unter anderem wird in diesem Kapitel das Konzept funktionaler Programmiersprachen behandelt, gefolgt von einer Einführung in die Grundlagen der Programmiersprache Elm. Im Anschluss werden in Kapitel \ref{sec:Evaluierung der Programmiersprache Elm} \glqq\nameref{sec:Evaluierung der Programmiersprache Elm}\grqq~Bewertungskriterien vorgestellt und erläutert, anhand derer die vorherigen Aussagen überprüft werden. Dazu gehören unter anderem die Evaluation der Performanz, die Klärung der Frage wie hilfreich der Elm-Compiler tatsächlich ist und ob mit der Verwendung von Elm Kompromisse bei der Kompatibilität mit unterschiedlichen Browsern gemacht werden müssen. Dem folgt die Dokumentation und Erläuterung des praktischen Teils dieser wissenschaftlichen Arbeit, in der zunächst einmal der geplante Programmablauf, sowie die genutzte Entwicklungsumgebung vorgestellt werden.  Letztlich mündet der Teil in einer Auswertung der aufgestellten Bewertungskriterien. Auf Grundlage der Auswertung werden im letzten Kapitel \ref{chap:fazit} \glqq\nameref{chap:fazit}\grqq~die Ergebnisse gegenübergestellt. Das Ziel dieser wissenschaftlichen Arbeit soll es sein, die Daseinsberechtigung der Programmiersprache Elm im Bereich der Frontend\footnote{Bezeichnet weitestgehend den Teil einer Webapplikation, der dem Nutzer dargestellt wird.}-Webentwicklung zu prüfen und eine Aussage darüber zu treffen, ob Elm die herkömmliche Entwicklung von Webapplikationen verbessert und tatsächlich eine valide Alternative darstellt.