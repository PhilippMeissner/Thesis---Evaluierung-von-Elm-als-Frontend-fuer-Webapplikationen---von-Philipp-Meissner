\chapter{Einleitung}
\label{sec:einleitung}
Die heutige Welt ist sehr schnelllebig und spielt sich immer mehr im Internet ab. Firmen präsentieren sich auf ihren Webseiten und akquirieren dadurch Neukunden. Ganze Geschäfte leben nur noch durch den Online-Handel und besitzen keinerlei Verkaufsläden, in denen ein Kunde das Produkt vorab in den Händen halten kann. Umso wichtiger ist es dann, das Produkt auf der Webseite außergewöhnlich gut zu präsentieren, um den Kunden zu überzeugen. Webseiten dieser und im Grunde jeglicher Art zielen immer darauf ab, einem Nutzer Informationen bereitzustellen und entsprechend angenehm zu präsentieren. Doch die Entwicklung solcher Systeme ist komplex und geht weit über das Design hinaus. Folglich ist ein großer Teil der Ausgaben von Firmen, die sich online präsentieren, die Bezahlung von Entwicklern für ihre Webapplikationen. Diese können simple Webseiten ohne großartige Funktionen sein, aber auch komplexe Systeme wie ein automatisierter Online-Handel, in dem die Nutzer ihren gesamten Einkauf abwickeln können, mitsamt Bezahlung. Demzufolge ist es für die Firmen von großem Interesse, dass die angestellten Entwickler zügig Ergebnisse in der Entwicklung der Webapplikationen machen. Damit der Entwickler effektiv arbeiten kann, braucht er ihn unterstützende Systeme, angefangen bei den Werkzeugen wie seiner Entwicklungsumgebung, bis hin zur tatsächlichen Programmiersprache. Diese wissenschaftliche Arbeit befasst sich mit der neuen Programmiersprache Elm, welche eine funktionale, den deklarativen Programmierparadigmen folgende, Programmiersprache ist. Sie wurde zu Beginn ihrer Entwicklung für die Entwicklung von grafischen Benutzeroberflächen und der Verbildlichung mathematischer Funktionen genutzt, bewegt sich nun jedoch immer weiter in Richtung der Webentwicklung und kommt mit einigen Neuheiten, Veränderungen und einer aktiven Gemeinschaft an Open-Source Entwicklern.\\
Elm verspricht eine blitzschnelle Darstellung von Inhalten, selbst bei riesigen Datenmengen mit Hilfe einer Technik die $virtual-dom$ genannt wird. Auch soll es keinerlei Laufzeitfehler mehr geben, da die gesamte Sprache mit Garantien ausgeschmückt ist, die im Zusammenspiel mit dem eigens entwickelten Compiler alle möglichen Fehler vorab findet und darauf hinweist.\\
All diese Versprechungen werden anhand mehrerer Bewertungskriterien während einer empirischen Entwicklung einer Webapplikation geprüft.