\chapter*{Abstract}
\label{sec:Abstract}
Motiviert durch die Versprechungen, die durch die funktionale Programmiersprache Elm gemacht werden \cite[Vgl.]{elm-no-runtime-errors}, ist das Ziel der vorliegenden Bachelor-Thesis, die Programmiersprache, welche in natives JavaScript kompiliert wird, für die Verwendung als Frontend für Webapplikationen anhand einer empirischen Untersuchung zu evaluieren.

Im Zuge dessen wurde eine \acl{SPA} in nativen Elm-Code überführt. Dabei wurde versucht weitestgehend die \acl{CSS}s und \acl{JS}-Dateien beizubehalten. Diese praktische Ausarbeitung wurde anhand unterschiedlichster Bewertungskriterien, wie beispielsweise die Zuverlässigkeit der Applikation, die Performanz einer Elm-Applikation und Portabilität zwischen unterschiedlichen Betriebssystemen und Browsern evaluiert. Ferner wurde die Dateigröße einer minimalen Elm-Applikation ausgewertet und anderen Frameworks gegenübergestellt. Darüber hinaus wurde der erzeugte Quellcode auf Wartbarkeit, Lesbarkeit, und Wiederverwendbarkeit untersucht.

Die praktische Ausarbeitung und Evaluation ergab, dass die Programmiersprache Elm die aufgestellten Kriterien in fast allen Punkten erfüllt. Lediglich die Interoperabilität von bestehenden \ac{JS}-Skripten in eine Elm-Applikation, sowie die Dateigröße einer minimalen Elm-Applikation erwiesen sich als nicht vollständig erfüllt. Die Interoperabilität war dahingehend nicht vollständig funktional, als dass es notwendig war Änderungen an den bestehenden \ac{JS}-Skripten anzufertigen, sowie die nicht praktikable Möglichkeit \ac{CSS}-Dateien nativ in Elm einzubinden. Der Elm-Compiler lieferte keinen Weg die Dateigröße der kompilierten Applikation zu minimieren. Stattdessen musste auf externe Werkzeuge zurückgegriffen werden. Abgesehen davon erwies sich eine Elm-Applikation als sehr performant, effizient und zuverlässig. Zusätzlich gibt es bereits eine Vielzahl an unterstützenden Werkzeugen zur Entwicklung mit Elm.

Die Ergebnisse erlauben die Schlussfolgerung, dass die Entwicklung mit Elm sehr empfehlenswert für Frontend-Entwickler ist. Das Typensystem in Elm erlaubt dem Elm-Compiler eine Vielzahl an Fehlern vorab zu finden. Ferner ist es für einen Entwickler nicht weiter notwendig die erzeugte Elm-Applikation auf allen Browsern separat testen zu müssen, da der Elm-Compiler die Applikation in natives \ac{JS} kompiliert, welches kompatibel mit dem ECMAScript-Standard ist. Daraus entsteht eine enorme Zeitersparnis, die gepaart mit einer effizienten Applikation und einem zuverlässigen Compiler einen deutlich positiven Eindruck von Elm hinterlässt.